% Options for packages loaded elsewhere
\PassOptionsToPackage{unicode}{hyperref}
\PassOptionsToPackage{hyphens}{url}
%
\documentclass[
]{article}
\usepackage{amsmath,amssymb}
\usepackage{lmodern}
\usepackage{ifxetex,ifluatex}
\ifnum 0\ifxetex 1\fi\ifluatex 1\fi=0 % if pdftex
  \usepackage[T1]{fontenc}
  \usepackage[utf8]{inputenc}
  \usepackage{textcomp} % provide euro and other symbols
\else % if luatex or xetex
  \usepackage{unicode-math}
  \defaultfontfeatures{Scale=MatchLowercase}
  \defaultfontfeatures[\rmfamily]{Ligatures=TeX,Scale=1}
\fi
% Use upquote if available, for straight quotes in verbatim environments
\IfFileExists{upquote.sty}{\usepackage{upquote}}{}
\IfFileExists{microtype.sty}{% use microtype if available
  \usepackage[]{microtype}
  \UseMicrotypeSet[protrusion]{basicmath} % disable protrusion for tt fonts
}{}
\makeatletter
\@ifundefined{KOMAClassName}{% if non-KOMA class
  \IfFileExists{parskip.sty}{%
    \usepackage{parskip}
  }{% else
    \setlength{\parindent}{0pt}
    \setlength{\parskip}{6pt plus 2pt minus 1pt}}
}{% if KOMA class
  \KOMAoptions{parskip=half}}
\makeatother
\usepackage{xcolor}
\IfFileExists{xurl.sty}{\usepackage{xurl}}{} % add URL line breaks if available
\IfFileExists{bookmark.sty}{\usepackage{bookmark}}{\usepackage{hyperref}}
\hypersetup{
  pdftitle={Econometría Avanzada},
  pdfauthor={Presentado por: Giovany Astroz y Javier leal},
  hidelinks,
  pdfcreator={LaTeX via pandoc}}
\urlstyle{same} % disable monospaced font for URLs
\usepackage[margin=1in]{geometry}
\usepackage{color}
\usepackage{fancyvrb}
\newcommand{\VerbBar}{|}
\newcommand{\VERB}{\Verb[commandchars=\\\{\}]}
\DefineVerbatimEnvironment{Highlighting}{Verbatim}{commandchars=\\\{\}}
% Add ',fontsize=\small' for more characters per line
\usepackage{framed}
\definecolor{shadecolor}{RGB}{248,248,248}
\newenvironment{Shaded}{\begin{snugshade}}{\end{snugshade}}
\newcommand{\AlertTok}[1]{\textcolor[rgb]{0.94,0.16,0.16}{#1}}
\newcommand{\AnnotationTok}[1]{\textcolor[rgb]{0.56,0.35,0.01}{\textbf{\textit{#1}}}}
\newcommand{\AttributeTok}[1]{\textcolor[rgb]{0.77,0.63,0.00}{#1}}
\newcommand{\BaseNTok}[1]{\textcolor[rgb]{0.00,0.00,0.81}{#1}}
\newcommand{\BuiltInTok}[1]{#1}
\newcommand{\CharTok}[1]{\textcolor[rgb]{0.31,0.60,0.02}{#1}}
\newcommand{\CommentTok}[1]{\textcolor[rgb]{0.56,0.35,0.01}{\textit{#1}}}
\newcommand{\CommentVarTok}[1]{\textcolor[rgb]{0.56,0.35,0.01}{\textbf{\textit{#1}}}}
\newcommand{\ConstantTok}[1]{\textcolor[rgb]{0.00,0.00,0.00}{#1}}
\newcommand{\ControlFlowTok}[1]{\textcolor[rgb]{0.13,0.29,0.53}{\textbf{#1}}}
\newcommand{\DataTypeTok}[1]{\textcolor[rgb]{0.13,0.29,0.53}{#1}}
\newcommand{\DecValTok}[1]{\textcolor[rgb]{0.00,0.00,0.81}{#1}}
\newcommand{\DocumentationTok}[1]{\textcolor[rgb]{0.56,0.35,0.01}{\textbf{\textit{#1}}}}
\newcommand{\ErrorTok}[1]{\textcolor[rgb]{0.64,0.00,0.00}{\textbf{#1}}}
\newcommand{\ExtensionTok}[1]{#1}
\newcommand{\FloatTok}[1]{\textcolor[rgb]{0.00,0.00,0.81}{#1}}
\newcommand{\FunctionTok}[1]{\textcolor[rgb]{0.00,0.00,0.00}{#1}}
\newcommand{\ImportTok}[1]{#1}
\newcommand{\InformationTok}[1]{\textcolor[rgb]{0.56,0.35,0.01}{\textbf{\textit{#1}}}}
\newcommand{\KeywordTok}[1]{\textcolor[rgb]{0.13,0.29,0.53}{\textbf{#1}}}
\newcommand{\NormalTok}[1]{#1}
\newcommand{\OperatorTok}[1]{\textcolor[rgb]{0.81,0.36,0.00}{\textbf{#1}}}
\newcommand{\OtherTok}[1]{\textcolor[rgb]{0.56,0.35,0.01}{#1}}
\newcommand{\PreprocessorTok}[1]{\textcolor[rgb]{0.56,0.35,0.01}{\textit{#1}}}
\newcommand{\RegionMarkerTok}[1]{#1}
\newcommand{\SpecialCharTok}[1]{\textcolor[rgb]{0.00,0.00,0.00}{#1}}
\newcommand{\SpecialStringTok}[1]{\textcolor[rgb]{0.31,0.60,0.02}{#1}}
\newcommand{\StringTok}[1]{\textcolor[rgb]{0.31,0.60,0.02}{#1}}
\newcommand{\VariableTok}[1]{\textcolor[rgb]{0.00,0.00,0.00}{#1}}
\newcommand{\VerbatimStringTok}[1]{\textcolor[rgb]{0.31,0.60,0.02}{#1}}
\newcommand{\WarningTok}[1]{\textcolor[rgb]{0.56,0.35,0.01}{\textbf{\textit{#1}}}}
\usepackage{graphicx}
\makeatletter
\def\maxwidth{\ifdim\Gin@nat@width>\linewidth\linewidth\else\Gin@nat@width\fi}
\def\maxheight{\ifdim\Gin@nat@height>\textheight\textheight\else\Gin@nat@height\fi}
\makeatother
% Scale images if necessary, so that they will not overflow the page
% margins by default, and it is still possible to overwrite the defaults
% using explicit options in \includegraphics[width, height, ...]{}
\setkeys{Gin}{width=\maxwidth,height=\maxheight,keepaspectratio}
% Set default figure placement to htbp
\makeatletter
\def\fps@figure{htbp}
\makeatother
\setlength{\emergencystretch}{3em} % prevent overfull lines
\providecommand{\tightlist}{%
  \setlength{\itemsep}{0pt}\setlength{\parskip}{0pt}}
\setcounter{secnumdepth}{-\maxdimen} % remove section numbering
\ifluatex
  \usepackage{selnolig}  % disable illegal ligatures
\fi

\title{Econometría Avanzada}
\usepackage{etoolbox}
\makeatletter
\providecommand{\subtitle}[1]{% add subtitle to \maketitle
  \apptocmd{\@title}{\par {\large #1 \par}}{}{}
}
\makeatother
\subtitle{Primer Examen Parcial}
\author{Presentado por: Giovany Astroz y Javier leal}
\date{12 marzo 2021}

\begin{document}
\maketitle

\hypertarget{estimaciuxf3n-de-la-demanda-de-cerveza}{%
\section{1. Estimación de la demanda de
cerveza}\label{estimaciuxf3n-de-la-demanda-de-cerveza}}

\hypertarget{anuxe1lisis-de-regresiuxf3n}{%
\subsection{1.1. Análisis de
regresión}\label{anuxe1lisis-de-regresiuxf3n}}

\hypertarget{a.-cree-una-nueva-variable-llamada-marca-que-contenga-la-marca-de-cada-upc-de-cerveza-por-ejemplo-heineken.}{%
\subsubsection{a. Cree una nueva variable llamada ``marca'' que contenga
la marca de cada UPC de cerveza (por ejemplo,
Heineken).}\label{a.-cree-una-nueva-variable-llamada-marca-que-contenga-la-marca-de-cada-upc-de-cerveza-por-ejemplo-heineken.}}

Al analizar la base de datos se encontró que la marca de cada UPC está
contenida dentro de la variable \emph{descrip}, sin embargo, dentro de
esta variable también se encuentran características adicionales que
hacen referencia principalmente a la variedad, presentación y contenido.
Ahora bien, estas especificaciones adicionales se encuentran siempre
luego de la marca, razón por la cual se decidió que una primera
aproximación para encontrar coincidencias sería extraer la primera
palabra de la descripción de cada UPC. Siguiendo esta aproximación se
encuentra que existen 172 palabras diferentes que clasifican \emph{a
priori} las 787 observaciones de la base de datos.

Claramente, este acercamiento no proporciona una clasificación perfecta,
pues existen marcas distintas que empiezan por la misma palabra. Por lo
tanto, se realizó una verificación manual para identificar las marcas
cuyos UPC eran correctamente agrupados bajo este criterio y las que no.
Durante este proceso se encontraron más problemas con los datos que se
mencionarán más adelante.

Luego de reconocer todos las limitaciones y correcciones que se debían
hacer, se creó la variable \emph{marca} de la siguiente manera:

\begin{enumerate}
\def\labelenumi{\arabic{enumi}.}
\tightlist
\item
  Se crea la variable \emph{marca} extrayendo la primera palabra de la
  variable \emph{descrip}. Aunque identificar la marca por la primera
  palabra no es suficiente en muchos casos, sí es una buena aproximación
  inicial.
\item
  Se diferencian las marcas que coinciden en la primera palabra. Por
  ejemplo, bajo la palabra ``SAMUEL'' quedaban agrupadas dos marcas
  diferentes: Samuel Smith y Samuel Adams.
\item
  Se unifican bajo el mismo nombre marcas que inicialmente quedaron
  separadas por omisión de caracteres especiales, abreviaturas o errores
  tipográficos en el nombre. Por ejemplo, la marca Murphy's Irish Stout
  se encontraba escrita con y sin apóstrofo: ``MURPHY`S'' y ``MURPHYS'';
  la marca Leinenkugel se escribía también bajo la abreviatura
  ``LEINNKGL'' y la marca Guinness se encontraba incorrectamente escrita
  como ``GUINESS''.
\item
  Se cambia la primera palabra por el nombre completo para todas las
  marcas con al menos dos observaciones y para aquellas con una sola
  observación en que se consideró necesario. Por ejemplo, se sustituye
  la palabra ``SIERRA'' por ``SIERRA NEVADA''. Si bien es cierto que
  estas marcas ya se encontraban correctamente identificadas, se
  prefirió poner el nombre completo para tener una base de datos más
  robusta.
\item
  Se eliminan de la base de datos los UPC que corresponden a marcas que
  no son de cerveza (vinos, sidra, bebidas sustitutas). Por ejemplo, se
  identifica que las variedades de la marca Sante (hardonnay, white,
  cabernet) corresponden a diferentes tipos de vino, y que la marca Hard
  Core produce sidra, una bebida alcohólica producto de la fermentación
  de la manzana. Específicante, se eliminaron de la base los UPC
  asociados a las marcas Sante, Ariel, Sutter Home, St Regis, Hard Core,
  Hornsby's, Santa, Zima, Woodchuck y Woodpecker.
\item
  En concordancia con el punto 1.1.c., se eliminan aquellos UPC para los
  cuales no es posible inferir la marca y además, aun ignorando el hecho
  de que no se conocía la marca, tampoco se podía determinar su
  ``nacionalidad''. Por ejemplo, la descripción para el UPC 1820000875
  es simplemente ``WINTER BREW'', la cual hace referencia a una variedad
  de cerveza que es común en varias marcas. A modo de ilustración, esta
  prodría pertenecer a la marca Pete's Wicked (americana) o a Fuller's
  (inglesa). Dado lo anterior, como tampocó se podía definir si se
  trataba de una cerveza nacional o importada se decidió eliminar esta
  observación. Así mismo, se eliminaron las observaciones con las
  descripciones ``WIT'', ``BEER LIMIT'', ``BEERS OF THE WORLD'' y ``OS
  CLASSIC'', las cuales no aportaban información suficiente para
  contruir las variables \emph{importada} y \emph{artesanal}.
\end{enumerate}

Es conveniente mencionar que existian casos que aplicaban a más de una
de las correcciones indicadas en los pasos 2, 3 y 4. Por ejemplo, bajo
la palabra ``SAM'' se encontraban dos marcas diferentes: Sam Smith y Sam
Adams (paso 2); precisamente, ``SAM SMITH'' y ``SAM ADAMS''
correspondían a la abreviatura de Samuel Smith y Samuel Adams (paso 3);
y finalmente, las marcas señalas contaban con nombres de dos palabras
(paso 4). En estos casos, la corrección se realizó en un solo paso y se
puede encontrar en cualquiera de estos.

Por otro lado, habían marcas con nombres diferentes pero pertenecientes
a la misma empresa en donde no era muy claro si debian tomarse como una
sola o no. Por ejemplo, las marcas Ice Draft Light y Bud Ice son
propiedad al igual que Budweiser de la multinacional Anheuser-Busch
InBev. A pesar de que las dos primeras compartían elementos de
identificación y construcción de marca con Budweiser (el vocablo Bud y
la referencia a Budweiser en la presentación del producto) se decidió
dejarlas como marcas diferentes. Para cada caso similar se tomó la
decisión de agrupar marcas o no con base en las estrategias de marketing
de las empresas y las diferencias en los productos mismos.

Finalmente, la base \emph{beer} contiene la nueva variable \emph{marca}
con todas las correciones mencionadas anteriormente, donde ahora se
tienen 753 observaciones y 150 marcas diferentes de cerveza.

\begin{Shaded}
\begin{Highlighting}[]
\FunctionTok{ls}\NormalTok{()}
\FunctionTok{remove}\NormalTok{(}\AttributeTok{list =} \FunctionTok{ls}\NormalTok{())}

\CommentTok{\# Cargando paquetes}
\FunctionTok{library}\NormalTok{(readr)}
\FunctionTok{library}\NormalTok{(rmarkdown)}
\FunctionTok{library}\NormalTok{(tidyverse)}
\FunctionTok{library}\NormalTok{(stringr)}
\FunctionTok{library}\NormalTok{(tibble)}
\FunctionTok{library}\NormalTok{(dplyr)}
\FunctionTok{library}\NormalTok{(tokenizers)}
\FunctionTok{library}\NormalTok{(writexl)}
\FunctionTok{library}\NormalTok{(expss)}
\FunctionTok{library}\NormalTok{(ggplot2)}
\FunctionTok{library}\NormalTok{(stargazer)}



\CommentTok{\# Definiendo directorio de trabajo}
\FunctionTok{setwd}\NormalTok{(}\StringTok{"\textasciitilde{}/Documents/Maestria/Primer semestre/Econometría Avanzada/Parcial 1"}\NormalTok{)}

\CommentTok{\# Cargando la base de datos}
\NormalTok{beer }\OtherTok{\textless{}{-}} \FunctionTok{read\_csv}\NormalTok{(}\StringTok{"beer.csv"}\NormalTok{)}

\CommentTok{\# Primera aproximación de clasificación utilizando la primera palabra de "descrip"}

\NormalTok{prim\_palabra\_descrip }\OtherTok{\textless{}{-}} \FunctionTok{word}\NormalTok{(beer}\SpecialCharTok{$}\NormalTok{descrip, }\DecValTok{1}\NormalTok{)}
\NormalTok{frec\_prim\_palabra }\OtherTok{\textless{}{-}} \FunctionTok{sort}\NormalTok{(}\FunctionTok{table}\NormalTok{(prim\_palabra\_descrip), }\AttributeTok{decreasing=}\NormalTok{T)}
\FunctionTok{head}\NormalTok{(frec\_prim\_palabra, }\DecValTok{10}\NormalTok{)}

\CommentTok{\#{-}{-}{-}{-}{-}{-}{-}{-}{-}{-}{-}{-}{-}{-}{-}{-}{-}{-}{-}{-}{-}{-}{-}{-}{-}{-}{-}{-}{-}{-}{-}\#}
\CommentTok{\#  Creando la variable *marca*  \#}
\CommentTok{\#{-}{-}{-}{-}{-}{-}{-}{-}{-}{-}{-}{-}{-}{-}{-}{-}{-}{-}{-}{-}{-}{-}{-}{-}{-}{-}{-}{-}{-}{-}{-}\#}

\CommentTok{\# 1. Extraer la primera palabra de la variable "descrip"}
\NormalTok{beer}\SpecialCharTok{$}\NormalTok{marca }\OtherTok{\textless{}{-}} \FunctionTok{word}\NormalTok{(beer}\SpecialCharTok{$}\NormalTok{descrip, }\DecValTok{1}\NormalTok{)}

\CommentTok{\# Correcciones:}

\CommentTok{\# 2. Marcas distintas que coinciden en la primera palabra}
\NormalTok{beer}\SpecialCharTok{$}\NormalTok{marca[}\FunctionTok{str\_detect}\NormalTok{(beer}\SpecialCharTok{$}\NormalTok{descrip, }\StringTok{"\^{}OLD STYLE"}\NormalTok{)]}\OtherTok{\textless{}{-}}\StringTok{"OLD STYLE"}
\NormalTok{beer}\SpecialCharTok{$}\NormalTok{marca[}\FunctionTok{str\_detect}\NormalTok{(beer}\SpecialCharTok{$}\NormalTok{descrip, }\StringTok{"\^{}OLD MILWAUKEE"}\NormalTok{)]}\OtherTok{\textless{}{-}}\StringTok{"OLD MILWAUKEE"}
\NormalTok{beer}\SpecialCharTok{$}\NormalTok{marca[}\FunctionTok{str\_detect}\NormalTok{(beer}\SpecialCharTok{$}\NormalTok{descrip, }\StringTok{"\^{}OLD ENGLISH"}\NormalTok{)]}\OtherTok{\textless{}{-}}\StringTok{"OLD ENGLISH"}
\NormalTok{beer}\SpecialCharTok{$}\NormalTok{marca[}\FunctionTok{str\_detect}\NormalTok{(beer}\SpecialCharTok{$}\NormalTok{descrip, }\StringTok{"\^{}SAMUEL SMITH"}\NormalTok{)]}\OtherTok{\textless{}{-}}\StringTok{"SAMUEL SMITH"}
\NormalTok{beer}\SpecialCharTok{$}\NormalTok{marca[}\FunctionTok{str\_detect}\NormalTok{(beer}\SpecialCharTok{$}\NormalTok{descrip, }\StringTok{"\^{}SAMUEL ADAMS"}\NormalTok{)]}\OtherTok{\textless{}{-}}\StringTok{"SAMUEL ADAMS"}
\NormalTok{beer}\SpecialCharTok{$}\NormalTok{marca[}\FunctionTok{str\_detect}\NormalTok{(beer}\SpecialCharTok{$}\NormalTok{descrip, }\StringTok{"\^{}SAM SMITH"}\NormalTok{)]}\OtherTok{\textless{}{-}}\StringTok{"SAMUEL SMITH"}
\NormalTok{beer}\SpecialCharTok{$}\NormalTok{marca[}\FunctionTok{str\_detect}\NormalTok{(beer}\SpecialCharTok{$}\NormalTok{descrip, }\StringTok{"\^{}SAM ADAMS"}\NormalTok{)]}\OtherTok{\textless{}{-}}\StringTok{"SAMUEL ADAMS"}
\NormalTok{beer}\SpecialCharTok{$}\NormalTok{marca[}\FunctionTok{str\_detect}\NormalTok{(beer}\SpecialCharTok{$}\NormalTok{descrip, }\StringTok{"\^{}ST REGIS"}\NormalTok{)]}\OtherTok{\textless{}{-}}\StringTok{"ST REGIS"}
\NormalTok{beer}\SpecialCharTok{$}\NormalTok{marca[}\FunctionTok{str\_detect}\NormalTok{(beer}\SpecialCharTok{$}\NormalTok{descrip, }\StringTok{"\^{}ST PAULI"}\NormalTok{)]}\OtherTok{\textless{}{-}}\StringTok{"ST PAULI GIRL"}
\NormalTok{beer}\SpecialCharTok{$}\NormalTok{marca[}\FunctionTok{str\_detect}\NormalTok{(beer}\SpecialCharTok{$}\NormalTok{descrip, }\StringTok{"\^{}BLUE MOON"}\NormalTok{)]}\OtherTok{\textless{}{-}}\StringTok{"BLUE MOON"}
\NormalTok{beer}\SpecialCharTok{$}\NormalTok{marca[}\FunctionTok{str\_detect}\NormalTok{(beer}\SpecialCharTok{$}\NormalTok{descrip, }\StringTok{"\^{}BLUE RIDGE"}\NormalTok{)]}\OtherTok{\textless{}{-}}\StringTok{"BLUE RIDGE"}
\NormalTok{beer}\SpecialCharTok{$}\NormalTok{marca[}\FunctionTok{str\_detect}\NormalTok{(beer}\SpecialCharTok{$}\NormalTok{descrip, }\StringTok{"\^{}RED WOLF"}\NormalTok{)]}\OtherTok{\textless{}{-}}\StringTok{"RED WOLF"}
\NormalTok{beer}\SpecialCharTok{$}\NormalTok{marca[}\FunctionTok{str\_detect}\NormalTok{(beer}\SpecialCharTok{$}\NormalTok{descrip, }\StringTok{"\^{}RED DOG"}\NormalTok{)]}\OtherTok{\textless{}{-}}\StringTok{"RED DOG"}
\NormalTok{beer}\SpecialCharTok{$}\NormalTok{marca[}\FunctionTok{str\_detect}\NormalTok{(beer}\SpecialCharTok{$}\NormalTok{descrip, }\StringTok{"\^{}RED STRIPE"}\NormalTok{)]}\OtherTok{\textless{}{-}}\StringTok{"RED STRIPE"}
\NormalTok{beer}\SpecialCharTok{$}\NormalTok{marca[}\FunctionTok{str\_detect}\NormalTok{(beer}\SpecialCharTok{$}\NormalTok{descrip, }\StringTok{"\^{}RED HOOK"}\NormalTok{)]}\OtherTok{\textless{}{-}}\StringTok{"RED HOOK"}
\NormalTok{beer}\SpecialCharTok{$}\NormalTok{marca[}\FunctionTok{str\_detect}\NormalTok{(beer}\SpecialCharTok{$}\NormalTok{descrip, }\StringTok{"\^{}RED RIVER"}\NormalTok{)]}\OtherTok{\textless{}{-}}\StringTok{"RED RIVER"}
\NormalTok{beer}\SpecialCharTok{$}\NormalTok{marca[}\FunctionTok{str\_detect}\NormalTok{(beer}\SpecialCharTok{$}\NormalTok{descrip, }\StringTok{"\^{}BEER LIMIT"}\NormalTok{)]}\OtherTok{\textless{}{-}}\StringTok{"BEER LIMIT"}
\NormalTok{beer}\SpecialCharTok{$}\NormalTok{marca[}\FunctionTok{str\_detect}\NormalTok{(beer}\SpecialCharTok{$}\NormalTok{descrip, }\StringTok{"\^{}BEER OF AMERICA"}\NormalTok{)]}\OtherTok{\textless{}{-}}\StringTok{"BEER OF AMERICA"}
\NormalTok{beer}\SpecialCharTok{$}\NormalTok{marca[}\FunctionTok{str\_detect}\NormalTok{(beer}\SpecialCharTok{$}\NormalTok{descrip, }\StringTok{"\^{}BEERS OF THE"}\NormalTok{)]}\OtherTok{\textless{}{-}}\StringTok{"BEERS OF THE WORLD"}
\NormalTok{beer}\SpecialCharTok{$}\NormalTok{marca[}\FunctionTok{str\_detect}\NormalTok{(beer}\SpecialCharTok{$}\NormalTok{descrip, }\StringTok{"\^{}BIG BEAR"}\NormalTok{)]}\OtherTok{\textless{}{-}}\StringTok{"BIG BEAR"}
\NormalTok{beer}\SpecialCharTok{$}\NormalTok{marca[}\FunctionTok{str\_detect}\NormalTok{(beer}\SpecialCharTok{$}\NormalTok{descrip, }\StringTok{"\^{}BIG SHOULDERS"}\NormalTok{)]}\OtherTok{\textless{}{-}}\StringTok{"BIG SHOULDERS"}

\CommentTok{\# 3. Misma marca con nombres diferentes}
\NormalTok{beer}\SpecialCharTok{$}\NormalTok{marca[}\FunctionTok{str\_detect}\NormalTok{(beer}\SpecialCharTok{$}\NormalTok{marca, }\StringTok{"\^{}PETES$"}\NormalTok{)]}\OtherTok{\textless{}{-}}\StringTok{"PETE\textquotesingle{}S WICKED"}
\NormalTok{beer}\SpecialCharTok{$}\NormalTok{marca[}\FunctionTok{str\_detect}\NormalTok{(beer}\SpecialCharTok{$}\NormalTok{marca, }\StringTok{"\^{}PETE\textquotesingle{}S$"}\NormalTok{)]}\OtherTok{\textless{}{-}}\StringTok{"PETE\textquotesingle{}S WICKED"}
\NormalTok{beer}\SpecialCharTok{$}\NormalTok{marca[}\FunctionTok{str\_detect}\NormalTok{(beer}\SpecialCharTok{$}\NormalTok{marca, }\StringTok{"\^{}MURPHYS$"}\NormalTok{)]}\OtherTok{\textless{}{-}}\StringTok{"MURPHY\textquotesingle{}S IRISH"}
\NormalTok{beer}\SpecialCharTok{$}\NormalTok{marca[}\FunctionTok{str\_detect}\NormalTok{(beer}\SpecialCharTok{$}\NormalTok{marca, }\StringTok{"\^{}MURPHY\textquotesingle{}S$"}\NormalTok{)]}\OtherTok{\textless{}{-}}\StringTok{"MURPHY\textquotesingle{}S IRISH"}
\NormalTok{beer}\SpecialCharTok{$}\NormalTok{marca[}\FunctionTok{str\_detect}\NormalTok{(beer}\SpecialCharTok{$}\NormalTok{marca, }\StringTok{"\^{}FOSTERS$"}\NormalTok{)]}\OtherTok{\textless{}{-}}\StringTok{"FOSTER\textquotesingle{}S LAGER"}
\NormalTok{beer}\SpecialCharTok{$}\NormalTok{marca[}\FunctionTok{str\_detect}\NormalTok{(beer}\SpecialCharTok{$}\NormalTok{marca, }\StringTok{"\^{}FOSTER\textquotesingle{}S$"}\NormalTok{)]}\OtherTok{\textless{}{-}}\StringTok{"FOSTER\textquotesingle{}S LAGER"}
\NormalTok{beer}\SpecialCharTok{$}\NormalTok{marca[}\FunctionTok{str\_detect}\NormalTok{(beer}\SpecialCharTok{$}\NormalTok{marca, }\StringTok{"\^{}GUINESS$"}\NormalTok{)]}\OtherTok{\textless{}{-}}\StringTok{"GUINNESS STOUT"}
\NormalTok{beer}\SpecialCharTok{$}\NormalTok{marca[}\FunctionTok{str\_detect}\NormalTok{(beer}\SpecialCharTok{$}\NormalTok{marca, }\StringTok{"\^{}GUINNESS$"}\NormalTok{)]}\OtherTok{\textless{}{-}}\StringTok{"GUINNESS STOUT"}
\NormalTok{beer}\SpecialCharTok{$}\NormalTok{marca[}\FunctionTok{str\_detect}\NormalTok{(beer}\SpecialCharTok{$}\NormalTok{marca, }\StringTok{"\^{}BECKS$"}\NormalTok{)]}\OtherTok{\textless{}{-}}\StringTok{"BECK\textquotesingle{}S"}
\NormalTok{beer}\SpecialCharTok{$}\NormalTok{marca[}\FunctionTok{str\_detect}\NormalTok{(beer}\SpecialCharTok{$}\NormalTok{marca, }\StringTok{"\^{}HACKER{-}PSCHORR$"}\NormalTok{)]}\OtherTok{\textless{}{-}}\StringTok{"HACKER PSCHORR"}
\NormalTok{beer}\SpecialCharTok{$}\NormalTok{marca[}\FunctionTok{str\_detect}\NormalTok{(beer}\SpecialCharTok{$}\NormalTok{marca, }\StringTok{"\^{}PSCHORR$"}\NormalTok{)]}\OtherTok{\textless{}{-}}\StringTok{"HACKER PSCHORR"}
\NormalTok{beer}\SpecialCharTok{$}\NormalTok{marca[}\FunctionTok{str\_detect}\NormalTok{(beer}\SpecialCharTok{$}\NormalTok{marca, }\StringTok{"\^{}STROHS$"}\NormalTok{)]}\OtherTok{\textless{}{-}}\StringTok{"STROH\textquotesingle{}S"}
\NormalTok{beer}\SpecialCharTok{$}\NormalTok{marca[}\FunctionTok{str\_detect}\NormalTok{(beer}\SpecialCharTok{$}\NormalTok{marca, }\StringTok{"\^{}FOSTERS$"}\NormalTok{)]}\OtherTok{\textless{}{-}}\StringTok{"FOSTER\textquotesingle{}S"}
\NormalTok{beer}\SpecialCharTok{$}\NormalTok{marca[}\FunctionTok{str\_detect}\NormalTok{(beer}\SpecialCharTok{$}\NormalTok{marca, }\StringTok{"\^{}LEINNKGL$"}\NormalTok{)]}\OtherTok{\textless{}{-}}\StringTok{"LEINENKUGEL"}
\NormalTok{beer}\SpecialCharTok{$}\NormalTok{marca[}\FunctionTok{str\_detect}\NormalTok{(beer}\SpecialCharTok{$}\NormalTok{marca, }\StringTok{"\^{}LABATT$"}\NormalTok{)]}\OtherTok{\textless{}{-}}\StringTok{"LABATTS"}
\NormalTok{beer}\SpecialCharTok{$}\NormalTok{marca[}\FunctionTok{str\_detect}\NormalTok{(beer}\SpecialCharTok{$}\NormalTok{marca, }\StringTok{"\^{}MICHEAL$"}\NormalTok{)]}\OtherTok{\textless{}{-}}\StringTok{"MICHAEL SHEA\textquotesingle{}S"}
\NormalTok{beer}\SpecialCharTok{$}\NormalTok{marca[}\FunctionTok{str\_detect}\NormalTok{(beer}\SpecialCharTok{$}\NormalTok{marca, }\StringTok{"\^{}MICHAEL$"}\NormalTok{)]}\OtherTok{\textless{}{-}}\StringTok{"MICHAEL SHEA\textquotesingle{}S"}
\NormalTok{beer}\SpecialCharTok{$}\NormalTok{marca[}\FunctionTok{str\_detect}\NormalTok{(beer}\SpecialCharTok{$}\NormalTok{marca, }\StringTok{"\^{}SHARPS$"}\NormalTok{)]}\OtherTok{\textless{}{-}}\StringTok{"MILLER"}
\NormalTok{beer}\SpecialCharTok{$}\NormalTok{marca[}\FunctionTok{str\_detect}\NormalTok{(beer}\SpecialCharTok{$}\NormalTok{marca, }\StringTok{"\^{}CORONITA$"}\NormalTok{)]}\OtherTok{\textless{}{-}}\StringTok{"CORONA"}
\NormalTok{beer}\SpecialCharTok{$}\NormalTok{marca[}\FunctionTok{str\_detect}\NormalTok{(beer}\SpecialCharTok{$}\NormalTok{marca, }\StringTok{"\^{}CHICAGO\textquotesingle{}S$"}\NormalTok{)]}\OtherTok{\textless{}{-}}\StringTok{"LEGACY"}

\CommentTok{\# 4. Marcas con nombres de más de una palabra}
\NormalTok{beer}\SpecialCharTok{$}\NormalTok{marca[}\FunctionTok{str\_detect}\NormalTok{(beer}\SpecialCharTok{$}\NormalTok{marca, }\StringTok{"\^{}SPECIAL$"}\NormalTok{)]}\OtherTok{\textless{}{-}}\StringTok{"SPECIAL EXPORT"}
\NormalTok{beer}\SpecialCharTok{$}\NormalTok{marca[}\FunctionTok{str\_detect}\NormalTok{(beer}\SpecialCharTok{$}\NormalTok{marca, }\StringTok{"\^{}GOOSE$"}\NormalTok{)]}\OtherTok{\textless{}{-}}\StringTok{"GOOSE ISLAND"}
\NormalTok{beer}\SpecialCharTok{$}\NormalTok{marca[}\FunctionTok{str\_detect}\NormalTok{(beer}\SpecialCharTok{$}\NormalTok{marca, }\StringTok{"\^{}MILWAUKEE\textquotesingle{}S$"}\NormalTok{)]}\OtherTok{\textless{}{-}}\StringTok{"MILWAUKEE\textquotesingle{}S BEST"}
\NormalTok{beer}\SpecialCharTok{$}\NormalTok{marca[}\FunctionTok{str\_detect}\NormalTok{(beer}\SpecialCharTok{$}\NormalTok{marca, }\StringTok{"\^{}CARLINGS$"}\NormalTok{)]}\OtherTok{\textless{}{-}}\StringTok{"CARLING BLACK LABEL"}
\NormalTok{beer}\SpecialCharTok{$}\NormalTok{marca[}\FunctionTok{str\_detect}\NormalTok{(beer}\SpecialCharTok{$}\NormalTok{marca, }\StringTok{"\^{}DOS$"}\NormalTok{)]}\OtherTok{\textless{}{-}}\StringTok{"DOS EQUIS"}
\NormalTok{beer}\SpecialCharTok{$}\NormalTok{marca[}\FunctionTok{str\_detect}\NormalTok{(beer}\SpecialCharTok{$}\NormalTok{marca, }\StringTok{"\^{}NEW$"}\NormalTok{)]}\OtherTok{\textless{}{-}}\StringTok{"NEW AMSTERDAM"}
\NormalTok{beer}\SpecialCharTok{$}\NormalTok{marca[}\FunctionTok{str\_detect}\NormalTok{(beer}\SpecialCharTok{$}\NormalTok{marca, }\StringTok{"\^{}ROLLING$"}\NormalTok{)]}\OtherTok{\textless{}{-}}\StringTok{"ROLLING ROCK"}
\NormalTok{beer}\SpecialCharTok{$}\NormalTok{marca[}\FunctionTok{str\_detect}\NormalTok{(beer}\SpecialCharTok{$}\NormalTok{marca, }\StringTok{"\^{}BLACK$"}\NormalTok{)]}\OtherTok{\textless{}{-}}\StringTok{"BLACK DOG"}
\NormalTok{beer}\SpecialCharTok{$}\NormalTok{marca[}\FunctionTok{str\_detect}\NormalTok{(beer}\SpecialCharTok{$}\NormalTok{marca, }\StringTok{"\^{}CARTA$"}\NormalTok{)]}\OtherTok{\textless{}{-}}\StringTok{"CARTA BLANCA"}
\NormalTok{beer}\SpecialCharTok{$}\NormalTok{marca[}\FunctionTok{str\_detect}\NormalTok{(beer}\SpecialCharTok{$}\NormalTok{marca, }\StringTok{"\^{}MEISTER$"}\NormalTok{)]}\OtherTok{\textless{}{-}}\StringTok{"MEISTER BRAU"}
\NormalTok{beer}\SpecialCharTok{$}\NormalTok{marca[}\FunctionTok{str\_detect}\NormalTok{(beer}\SpecialCharTok{$}\NormalTok{marca, }\StringTok{"\^{}NAKED$"}\NormalTok{)]}\OtherTok{\textless{}{-}}\StringTok{"NAKED ASPEN"}
\NormalTok{beer}\SpecialCharTok{$}\NormalTok{marca[}\FunctionTok{str\_detect}\NormalTok{(beer}\SpecialCharTok{$}\NormalTok{marca, }\StringTok{"\^{}RHINO$"}\NormalTok{)]}\OtherTok{\textless{}{-}}\StringTok{"RHINO CHASERS"}
\NormalTok{beer}\SpecialCharTok{$}\NormalTok{marca[}\FunctionTok{str\_detect}\NormalTok{(beer}\SpecialCharTok{$}\NormalTok{marca, }\StringTok{"\^{}SIERRA$"}\NormalTok{)]}\OtherTok{\textless{}{-}}\StringTok{"SIERRA NEVADA"}
\NormalTok{beer}\SpecialCharTok{$}\NormalTok{marca[}\FunctionTok{str\_detect}\NormalTok{(beer}\SpecialCharTok{$}\NormalTok{marca, }\StringTok{"\^{}STATE$"}\NormalTok{)]}\OtherTok{\textless{}{-}}\StringTok{"STATE STREET"}
\NormalTok{beer}\SpecialCharTok{$}\NormalTok{marca[}\FunctionTok{str\_detect}\NormalTok{(beer}\SpecialCharTok{$}\NormalTok{marca, }\StringTok{"\^{}COLD$"}\NormalTok{)]}\OtherTok{\textless{}{-}}\StringTok{"COLD SPRING"}
\NormalTok{beer}\SpecialCharTok{$}\NormalTok{marca[}\FunctionTok{str\_detect}\NormalTok{(beer}\SpecialCharTok{$}\NormalTok{marca, }\StringTok{"\^{}COLT$"}\NormalTok{)]}\OtherTok{\textless{}{-}}\StringTok{"COLT 45"}
\NormalTok{beer}\SpecialCharTok{$}\NormalTok{marca[}\FunctionTok{str\_detect}\NormalTok{(beer}\SpecialCharTok{$}\NormalTok{marca, }\StringTok{"\^{}ELK$"}\NormalTok{)]}\OtherTok{\textless{}{-}}\StringTok{"ELK MOUNTAIN"}
\NormalTok{beer}\SpecialCharTok{$}\NormalTok{marca[}\FunctionTok{str\_detect}\NormalTok{(beer}\SpecialCharTok{$}\NormalTok{marca, }\StringTok{"\^{}HACKER$"}\NormalTok{)]}\OtherTok{\textless{}{-}}\StringTok{"HACKER PSCHORR"}
\NormalTok{beer}\SpecialCharTok{$}\NormalTok{marca[}\FunctionTok{str\_detect}\NormalTok{(beer}\SpecialCharTok{$}\NormalTok{marca, }\StringTok{"\^{}HARD$"}\NormalTok{)]}\OtherTok{\textless{}{-}}\StringTok{"HARD CORE"}
\NormalTok{beer}\SpecialCharTok{$}\NormalTok{marca[}\FunctionTok{str\_detect}\NormalTok{(beer}\SpecialCharTok{$}\NormalTok{marca, }\StringTok{"\^{}LITTLE$"}\NormalTok{)]}\OtherTok{\textless{}{-}}\StringTok{"LITTLE KINGS"}
\NormalTok{beer}\SpecialCharTok{$}\NormalTok{marca[}\FunctionTok{str\_detect}\NormalTok{(beer}\SpecialCharTok{$}\NormalTok{marca, }\StringTok{"\^{}MCEWANS$"}\NormalTok{)]}\OtherTok{\textless{}{-}}\StringTok{"MCEWANS SCOTCH"}
\NormalTok{beer}\SpecialCharTok{$}\NormalTok{marca[}\FunctionTok{str\_detect}\NormalTok{(beer}\SpecialCharTok{$}\NormalTok{marca, }\StringTok{"\^{}OS$"}\NormalTok{)]}\OtherTok{\textless{}{-}}\StringTok{"OS CLASSIC"}
\NormalTok{beer}\SpecialCharTok{$}\NormalTok{marca[}\FunctionTok{str\_detect}\NormalTok{(beer}\SpecialCharTok{$}\NormalTok{marca, }\StringTok{"\^{}PILSNER$"}\NormalTok{)]}\OtherTok{\textless{}{-}}\StringTok{"PILSNER URQUELL"}
\NormalTok{beer}\SpecialCharTok{$}\NormalTok{marca[}\FunctionTok{str\_detect}\NormalTok{(beer}\SpecialCharTok{$}\NormalTok{marca, }\StringTok{"\^{}WHEAT$"}\NormalTok{)]}\OtherTok{\textless{}{-}}\StringTok{"WHEAT HOOK"}
\NormalTok{beer}\SpecialCharTok{$}\NormalTok{marca[}\FunctionTok{str\_detect}\NormalTok{(beer}\SpecialCharTok{$}\NormalTok{marca, }\StringTok{"\^{}ICE$"}\NormalTok{)]}\OtherTok{\textless{}{-}}\StringTok{"ICE DRAFT LIGHT"}
\NormalTok{beer}\SpecialCharTok{$}\NormalTok{marca[}\FunctionTok{str\_detect}\NormalTok{(beer}\SpecialCharTok{$}\NormalTok{marca, }\StringTok{"\^{}O$"}\NormalTok{)]}\OtherTok{\textless{}{-}}\StringTok{"ROYAL GUARD"}
\NormalTok{beer}\SpecialCharTok{$}\NormalTok{marca[}\FunctionTok{str\_detect}\NormalTok{(beer}\SpecialCharTok{$}\NormalTok{marca, }\StringTok{"\^{}JOHN$"}\NormalTok{)]}\OtherTok{\textless{}{-}}\StringTok{"JOHN COURAGE"}
\NormalTok{beer}\SpecialCharTok{$}\NormalTok{marca[}\FunctionTok{str\_detect}\NormalTok{(beer}\SpecialCharTok{$}\NormalTok{marca, }\StringTok{"\^{}JW$"}\NormalTok{)]}\OtherTok{\textless{}{-}}\StringTok{"JW DUNDEE\textquotesingle{}S"}

\CommentTok{\# 5. Marcas de vinos y sidra}
\NormalTok{beer}\SpecialCharTok{$}\NormalTok{num }\OtherTok{\textless{}{-}} \FunctionTok{c}\NormalTok{(}\DecValTok{1}\SpecialCharTok{:}\DecValTok{787}\NormalTok{) }
\CommentTok{\#se crea esta variable para identificar las observaciones de las marcas que se van a eliminar}

\NormalTok{beer }\OtherTok{\textless{}{-}}\NormalTok{ beer[}\SpecialCharTok{{-}}\FunctionTok{c}\NormalTok{(}\DecValTok{1}\NormalTok{, }\DecValTok{22}\NormalTok{, }\DecValTok{23}\NormalTok{, }\DecValTok{74}\NormalTok{, }\DecValTok{122}\NormalTok{, }\DecValTok{123}\NormalTok{, }\DecValTok{124}\NormalTok{, }\DecValTok{125}\NormalTok{, }\DecValTok{250}\NormalTok{, }\DecValTok{319}\NormalTok{, }\DecValTok{383}\NormalTok{, }\DecValTok{384}\NormalTok{, }\DecValTok{385}\NormalTok{, }\DecValTok{386}\NormalTok{,}
                \DecValTok{476}\NormalTok{, }\DecValTok{477}\NormalTok{, }\DecValTok{557}\NormalTok{, }\DecValTok{558}\NormalTok{, }\DecValTok{604}\NormalTok{, }\DecValTok{605}\NormalTok{, }\DecValTok{606}\NormalTok{, }\DecValTok{626}\NormalTok{, }\DecValTok{627}\NormalTok{, }\DecValTok{628}\NormalTok{, }\DecValTok{629}\NormalTok{, }\DecValTok{630}\NormalTok{, }\DecValTok{631}\NormalTok{, }
                \DecValTok{678}\NormalTok{, }\DecValTok{679}\NormalTok{, }\DecValTok{703}\NormalTok{, }\DecValTok{704}\NormalTok{, }\DecValTok{705}\NormalTok{, }\DecValTok{706}\NormalTok{, }\DecValTok{750}\NormalTok{), ] }

\CommentTok{\# Identificando cuántas marcas de cervezas quedan}
\NormalTok{brands }\OtherTok{\textless{}{-}} \FunctionTok{sort}\NormalTok{(}\FunctionTok{table}\NormalTok{(beer}\SpecialCharTok{$}\NormalTok{marca), }\AttributeTok{decreasing =} \ConstantTok{TRUE}\NormalTok{)}
\NormalTok{brands}
\end{Highlighting}
\end{Shaded}

\hypertarget{b.-reporte-estaduxedsticas-descriptivas-del-valor-de-las-ventas-el-nuxfamero-de-unidades-vendidas-y-el-precio-promedio-por-marca-en-una-tabla.-comente-lo-que-encuentra.}{%
\subsubsection{b. Reporte estadísticas descriptivas del valor de las
ventas, el número de unidades vendidas y el precio promedio por marca en
una tabla. Comente lo que
encuentra.}\label{b.-reporte-estaduxedsticas-descriptivas-del-valor-de-las-ventas-el-nuxfamero-de-unidades-vendidas-y-el-precio-promedio-por-marca-en-una-tabla.-comente-lo-que-encuentra.}}

\begin{Shaded}
\begin{Highlighting}[]
\NormalTok{tab1 }\OtherTok{\textless{}{-}}\NormalTok{beer }\SpecialCharTok{\%\textgreater{}\%}    \FunctionTok{group\_by}\NormalTok{(marca) }\SpecialCharTok{\%\textgreater{}\%}    \FunctionTok{summarise}\NormalTok{(}\FunctionTok{mean}\NormalTok{(sales),             }\FunctionTok{median}\NormalTok{(sales),             }\FunctionTok{sd}\NormalTok{(sales))}
\NormalTok{tab2 }\OtherTok{\textless{}{-}}\NormalTok{beer }\SpecialCharTok{\%\textgreater{}\%}    \FunctionTok{group\_by}\NormalTok{(marca) }\SpecialCharTok{\%\textgreater{}\%}    \FunctionTok{summarise}\NormalTok{(}\FunctionTok{mean}\NormalTok{(price),             }\FunctionTok{median}\NormalTok{(price),             }\FunctionTok{sd}\NormalTok{(price))}
\NormalTok{tab3 }\OtherTok{\textless{}{-}}\NormalTok{ beer }\SpecialCharTok{\%\textgreater{}\%}    \FunctionTok{group\_by}\NormalTok{(marca) }\SpecialCharTok{\%\textgreater{}\%}    \FunctionTok{summarise}\NormalTok{(}\FunctionTok{mean}\NormalTok{(q),             }\FunctionTok{median}\NormalTok{(q),             }\FunctionTok{sd}\NormalTok{(q))}

\NormalTok{temporal12 }\OtherTok{=} \FunctionTok{merge}\NormalTok{(}\AttributeTok{by=}\StringTok{"marca"}\NormalTok{, tab1,tab2)}
\NormalTok{Tabla1.b }\OtherTok{=} \FunctionTok{merge}\NormalTok{ (}\AttributeTok{by=}\StringTok{"marca"}\NormalTok{, temporal12,tab3 )}
\end{Highlighting}
\end{Shaded}

\hypertarget{c.-haga-una-buxfasqueda-sobre-las-marcas-incluidas-en-la-lista-y-construya-las-siguientes-variables-dummy-1-importada-variable-igual-a-1-si-la-cerveza-es-originaria-de-un-pauxeds-distinto-a-los-estados-unidos-y-cero-en-otro-caso-y-2-artesanal-variable-igual-a-1-si-la-cerveza-es-producida-artesanalmente-y-cero-en-otro-caso.-reporte-estaduxedsticas-descriptivas-por-caracteruxedstica-en-una-tabla.}{%
\subsubsection{c.~Haga una búsqueda sobre las marcas incluidas en la
lista y construya las siguientes variables dummy: 1) ``importada'':
variable igual a 1 si la cerveza es originaria de un país distinto a los
Estados Unidos y cero en otro caso; y 2) ``artesanal'': variable igual a
1 si la cerveza es producida artesanalmente y cero en otro caso. Reporte
estadísticas descriptivas, por característica, en una
tabla.}\label{c.-haga-una-buxfasqueda-sobre-las-marcas-incluidas-en-la-lista-y-construya-las-siguientes-variables-dummy-1-importada-variable-igual-a-1-si-la-cerveza-es-originaria-de-un-pauxeds-distinto-a-los-estados-unidos-y-cero-en-otro-caso-y-2-artesanal-variable-igual-a-1-si-la-cerveza-es-producida-artesanalmente-y-cero-en-otro-caso.-reporte-estaduxedsticas-descriptivas-por-caracteruxedstica-en-una-tabla.}}

En primer lugar, con respecto a la variable \emph{importada}, es fácil
distinguir si una cerveza es nacional o importada cuando proviene de una
empresa pequeña o mediana con una localización definida. Sin embargo, la
industria de la cerveza ha crecido ampliamente durante las últimas
décadas, y para grandes multinaciones tal vez no es claro distinguir
esta característica. El aumento de la demanda a nivel mundial ha
desencadenado en una serie de decisiones logísticas por parte de las
multinacionales encaminadas a reducir costos a través de la producción
local, razón por la cual una misma marca puede tener fabricas en muchos
países. Por ejemplo, la marca de cerveza mexicana Corona decidió en el
2019 expandir su producción por fuera de su país estableciendoce en
otros lugares como China; Corona está aumentando la disponibilidad en
otros mercados y al mismo tiempo está reduciendo sus costos.

Ahora bien, al analizar un país con una demanda tan significativa como
lo es Estados Unidos, resulta natural pensar que muchas marcas
extranjeras prefieran elaborar su producción destinada al público
norteamericano en Estados Unidos. Para estos casos se decidió que se
tomará una marca como importada siempre y cuando su casa matriz se
encuentre fuera de Estados Unidos, incluso si la cerveza para este país
es fabricada ahí mismo. Por ejemplo, la marca de cerveza St.~Pauli Girl
es de origen alemán y su casa matriz se encuentra en la ciudad de
Bremen, Alemania. Aun cuando esta cerveza en Estados Unidos es fabricada
en la ciudad de San Luis, en Misuri, se tomó como importada. Además,
resulta lógico definir esta variable así puesto que la estrategia de
marketing de muchas de estas empresas con origen extranjero que tienen
sedes en EE. UU. consiste en resaltar su origen.

En segundo lugar, el criterio para clasificar una cerveza como artesanal
o de producción masiva está mucho menos claro. La Asociación americana
de cerveceros define a las cervecerías artesanales como aquellas que
cumplen con 3 características: son pequeñas, independientes y
tradicionales. El límite para el tamaño de una cervecería está
comunmente definido por el número anual de barriles producidos o la
participación en el mercado nacional (6 millones de barriles o 3\% de la
producción nacional). Por su parte, la independencia hace referencia al
porcentaje máximo de la empresa de la que puede ser dueño un actor que
no pertenezca a la industria en cuestión (máximo 25\%), y la tradición
se refiere a los ingredientes y recetas originales (Brewers Association,
s.f.).

Desde luego, la definición de cerveza artesanal no es única y existen
más aproximaciones a este concepto que suponen más o menos restricciones
sobre el producto. Además, no cabe duda que esta clasificación se hace
más imprecisa con los grandes cambios y la expansión que ha sufrido la
industria cervecera. Para ilustrar esto, marcas como Guinness y
Löwenbräu conservan su receta original prácticamente sin variaciones
desde su creación pero fueron incorporadas a grupos multinacionales más
grandes, por lo que ya no es claro si pierden o no su calidad de
artesanal. Además, también se suma la inexactitud de buscar esta
información, la cual muchas veces es escasa, 20 años después del momento
en el que fueron recolectados los datos, pues muchas cosas pudieron
suceder en este periodo que cambiaran la percepción de estas marcas.

En general, teniendo en cuenta que la construcción de esta variable está
sesgada por problemas de definición y medición, se utilizó la
información a nuestro alcance para identificar elementos en las cervezas
entre los anteriormente mencionados y poder clasificar las 150 marcas
que se encuentran en la base de datos.

\begin{Shaded}
\begin{Highlighting}[]
\CommentTok{\#{-}{-}{-}{-}{-}{-}{-}{-}{-}{-}{-}{-}{-}{-}{-}{-}{-}{-}{-}{-}{-}{-}{-}{-}{-}{-}{-}{-}{-}{-}{-}{-}{-}{-}{-}{-}{-}{-}{-}{-}{-}{-}{-}{-}{-}{-}{-}{-}{-}{-}{-}\#}
\CommentTok{\#  Creando las variables *importada* y *artesanal*  \#}
\CommentTok{\#{-}{-}{-}{-}{-}{-}{-}{-}{-}{-}{-}{-}{-}{-}{-}{-}{-}{-}{-}{-}{-}{-}{-}{-}{-}{-}{-}{-}{-}{-}{-}{-}{-}{-}{-}{-}{-}{-}{-}{-}{-}{-}{-}{-}{-}{-}{-}{-}{-}{-}{-}\#}

\CommentTok{\# Variable *importada*}
\NormalTok{beer}\SpecialCharTok{$}\NormalTok{importada }\OtherTok{\textless{}{-}} \FunctionTok{ifelse}\NormalTok{( beer}\SpecialCharTok{$}\NormalTok{marca }\SpecialCharTok{==} \StringTok{"BECK\textquotesingle{}S"} \SpecialCharTok{|}\NormalTok{ beer}\SpecialCharTok{$}\NormalTok{marca }\SpecialCharTok{==} \StringTok{"MOLSON"} \SpecialCharTok{|}
\NormalTok{                          beer}\SpecialCharTok{$}\NormalTok{marca }\SpecialCharTok{==} \StringTok{"LOWENBRAU"} \SpecialCharTok{|}\NormalTok{ beer}\SpecialCharTok{$}\NormalTok{marca }\SpecialCharTok{==} \StringTok{"TECATE"} \SpecialCharTok{|}
\NormalTok{                          beer}\SpecialCharTok{$}\NormalTok{marca }\SpecialCharTok{==} \StringTok{"FOSTER\textquotesingle{}S LAGER"} \SpecialCharTok{|}\NormalTok{ beer}\SpecialCharTok{$}\NormalTok{marca }\SpecialCharTok{==} \StringTok{"LABATTS"} \SpecialCharTok{|}
\NormalTok{                          beer}\SpecialCharTok{$}\NormalTok{marca }\SpecialCharTok{==} \StringTok{"MURPHY\textquotesingle{}S IRISH"} \SpecialCharTok{|}\NormalTok{ beer}\SpecialCharTok{$}\NormalTok{marca }\SpecialCharTok{==} \StringTok{"ST PAULI GIRL"} \SpecialCharTok{|}
\NormalTok{                          beer}\SpecialCharTok{$}\NormalTok{marca }\SpecialCharTok{==} \StringTok{"CARLING BLACK LABEL"} \SpecialCharTok{|}\NormalTok{ beer}\SpecialCharTok{$}\NormalTok{marca }\SpecialCharTok{==} \StringTok{"CORONA"} \SpecialCharTok{|}
\NormalTok{                          beer}\SpecialCharTok{$}\NormalTok{marca }\SpecialCharTok{==} \StringTok{"DOS EQUIS"} \SpecialCharTok{|}\NormalTok{ beer}\SpecialCharTok{$}\NormalTok{marca }\SpecialCharTok{==} \StringTok{"GUINNESS STOUT"} \SpecialCharTok{|} 
\NormalTok{                          beer}\SpecialCharTok{$}\NormalTok{marca }\SpecialCharTok{==} \StringTok{"HEINEKEN"} \SpecialCharTok{|}\NormalTok{ beer}\SpecialCharTok{$}\NormalTok{marca }\SpecialCharTok{==} \StringTok{"WARSTEINER"} \SpecialCharTok{|} 
\NormalTok{                          beer}\SpecialCharTok{$}\NormalTok{marca }\SpecialCharTok{==} \StringTok{"AMSTEL"} \SpecialCharTok{|}\NormalTok{ beer}\SpecialCharTok{$}\NormalTok{marca }\SpecialCharTok{==} \StringTok{"MOOSEHEAD"} \SpecialCharTok{|} 
\NormalTok{                          beer}\SpecialCharTok{$}\NormalTok{marca }\SpecialCharTok{==} \StringTok{"BASS"} \SpecialCharTok{|}\NormalTok{ beer}\SpecialCharTok{$}\NormalTok{marca }\SpecialCharTok{==} \StringTok{"CARTA BLANCA"} \SpecialCharTok{|} 
\NormalTok{                          beer}\SpecialCharTok{$}\NormalTok{marca }\SpecialCharTok{==} \StringTok{"HACKER PSCHORR"} \SpecialCharTok{|}\NormalTok{ beer}\SpecialCharTok{$}\NormalTok{marca }\SpecialCharTok{==} \StringTok{"KIRIN"} \SpecialCharTok{|} 
\NormalTok{                          beer}\SpecialCharTok{$}\NormalTok{marca }\SpecialCharTok{==} \StringTok{"MODELO"} \SpecialCharTok{|}\NormalTok{ beer}\SpecialCharTok{$}\NormalTok{marca }\SpecialCharTok{==} \StringTok{"MORETTI"} \SpecialCharTok{|} 
\NormalTok{                          beer}\SpecialCharTok{$}\NormalTok{marca }\SpecialCharTok{==} \StringTok{"SAMUEL SMITH"} \SpecialCharTok{|}\NormalTok{ beer}\SpecialCharTok{$}\NormalTok{marca }\SpecialCharTok{==} \StringTok{"SAPPORO"} \SpecialCharTok{|} 
\NormalTok{                          beer}\SpecialCharTok{$}\NormalTok{marca }\SpecialCharTok{==} \StringTok{"SPATEN"} \SpecialCharTok{|}\NormalTok{ beer}\SpecialCharTok{$}\NormalTok{marca }\SpecialCharTok{==} \StringTok{"ASAHI"} \SpecialCharTok{|} 
\NormalTok{                          beer}\SpecialCharTok{$}\NormalTok{marca }\SpecialCharTok{==} \StringTok{"CLAUSTHALER"} \SpecialCharTok{|}\NormalTok{ beer}\SpecialCharTok{$}\NormalTok{marca }\SpecialCharTok{==} \StringTok{"ELEPHANT"} \SpecialCharTok{|} 
\NormalTok{                          beer}\SpecialCharTok{$}\NormalTok{marca }\SpecialCharTok{==} \StringTok{"GROLSCH"} \SpecialCharTok{|}\NormalTok{ beer}\SpecialCharTok{$}\NormalTok{marca }\SpecialCharTok{==} \StringTok{"HARP"} \SpecialCharTok{|}   
\NormalTok{                          beer}\SpecialCharTok{$}\NormalTok{marca }\SpecialCharTok{==} \StringTok{"MCEWANS SCOTCH"} \SpecialCharTok{|}\NormalTok{ beer}\SpecialCharTok{$}\NormalTok{marca }\SpecialCharTok{==} \StringTok{"PILSNER URQUELL"} \SpecialCharTok{|}
\NormalTok{                          beer}\SpecialCharTok{$}\NormalTok{marca }\SpecialCharTok{==} \StringTok{"STEINLAGERS"} \SpecialCharTok{|}\NormalTok{ beer}\SpecialCharTok{$}\NormalTok{marca }\SpecialCharTok{==} \StringTok{"TENNENTS"} \SpecialCharTok{|}  
\NormalTok{                          beer}\SpecialCharTok{$}\NormalTok{marca }\SpecialCharTok{==} \StringTok{"TOSELLI"} \SpecialCharTok{|}\NormalTok{ beer}\SpecialCharTok{$}\NormalTok{marca }\SpecialCharTok{==} \StringTok{"BOHEMIA"} \SpecialCharTok{|}  
\NormalTok{                          beer}\SpecialCharTok{$}\NormalTok{marca }\SpecialCharTok{==} \StringTok{"BUCKLER"} \SpecialCharTok{|}\NormalTok{ beer}\SpecialCharTok{$}\NormalTok{marca }\SpecialCharTok{==} \StringTok{"CARLSBERG"} \SpecialCharTok{|}  
\NormalTok{                          beer}\SpecialCharTok{$}\NormalTok{marca }\SpecialCharTok{==} \StringTok{"CASTLEMAINE"} \SpecialCharTok{|}\NormalTok{ beer}\SpecialCharTok{$}\NormalTok{marca }\SpecialCharTok{==} \StringTok{"COOPERS"} \SpecialCharTok{|}  
\NormalTok{                          beer}\SpecialCharTok{$}\NormalTok{marca }\SpecialCharTok{==} \StringTok{"COUNTRY"} \SpecialCharTok{|}\NormalTok{ beer}\SpecialCharTok{$}\NormalTok{marca }\SpecialCharTok{==} \StringTok{"DOM"} \SpecialCharTok{|}  
\NormalTok{                          beer}\SpecialCharTok{$}\NormalTok{marca }\SpecialCharTok{==} \StringTok{"DORTMUNDER"} \SpecialCharTok{|}\NormalTok{ beer}\SpecialCharTok{$}\NormalTok{marca }\SpecialCharTok{==} \StringTok{"DUVEL"} \SpecialCharTok{|}  
\NormalTok{                          beer}\SpecialCharTok{$}\NormalTok{marca }\SpecialCharTok{==} \StringTok{"FRANZISKANER"} \SpecialCharTok{|}\NormalTok{ beer}\SpecialCharTok{$}\NormalTok{marca }\SpecialCharTok{==} \StringTok{"FULLERS"} \SpecialCharTok{|}  
\NormalTok{                          beer}\SpecialCharTok{$}\NormalTok{marca }\SpecialCharTok{==} \StringTok{"HAACK{-}BECK"} \SpecialCharTok{|}\NormalTok{ beer}\SpecialCharTok{$}\NormalTok{marca }\SpecialCharTok{==} \StringTok{"JOHN COURAGE"} \SpecialCharTok{|}  
\NormalTok{                          beer}\SpecialCharTok{$}\NormalTok{marca }\SpecialCharTok{==} \StringTok{"LEZAJSK"} \SpecialCharTok{|}\NormalTok{ beer}\SpecialCharTok{$}\NormalTok{marca }\SpecialCharTok{==} \StringTok{"MACKESON"} \SpecialCharTok{|}  
\NormalTok{                          beer}\SpecialCharTok{$}\NormalTok{marca }\SpecialCharTok{==} \StringTok{"MOUSSY"} \SpecialCharTok{|}\NormalTok{ beer}\SpecialCharTok{$}\NormalTok{marca }\SpecialCharTok{==} \StringTok{"NEWCASTLE"} \SpecialCharTok{|}  
\NormalTok{                          beer}\SpecialCharTok{$}\NormalTok{marca }\SpecialCharTok{==} \StringTok{"O.B."} \SpecialCharTok{|}\NormalTok{ beer}\SpecialCharTok{$}\NormalTok{marca }\SpecialCharTok{==} \StringTok{"OKOCIM"} \SpecialCharTok{|}    
\NormalTok{                          beer}\SpecialCharTok{$}\NormalTok{marca }\SpecialCharTok{==} \StringTok{"PACIFICO"} \SpecialCharTok{|}\NormalTok{ beer}\SpecialCharTok{$}\NormalTok{marca }\SpecialCharTok{==} \StringTok{"PERONI"} \SpecialCharTok{|}    
\NormalTok{                          beer}\SpecialCharTok{$}\NormalTok{marca }\SpecialCharTok{==} \StringTok{"RED STRIPE"} \SpecialCharTok{|}\NormalTok{ beer}\SpecialCharTok{$}\NormalTok{marca }\SpecialCharTok{==} \StringTok{"SUNTORY"} \SpecialCharTok{|}    
\NormalTok{                          beer}\SpecialCharTok{$}\NormalTok{marca }\SpecialCharTok{==} \StringTok{"TSINGTAO"} \SpecialCharTok{|}\NormalTok{ beer}\SpecialCharTok{$}\NormalTok{marca }\SpecialCharTok{==} \StringTok{"WHITBREAD"} \SpecialCharTok{|}      
\NormalTok{                          beer}\SpecialCharTok{$}\NormalTok{marca }\SpecialCharTok{==} \StringTok{"YOUNGS"} \SpecialCharTok{|}\NormalTok{ beer}\SpecialCharTok{$}\NormalTok{marca }\SpecialCharTok{==} \StringTok{"ZYWIEC"} \SpecialCharTok{|}        
\NormalTok{                          beer}\SpecialCharTok{$}\NormalTok{marca }\SpecialCharTok{==} \StringTok{"ROYAL GUARD"}\NormalTok{ , }\DecValTok{1}\NormalTok{, }\DecValTok{0}\NormalTok{)}
           
\CommentTok{\# Variable *artesanal*}
\NormalTok{beer}\SpecialCharTok{$}\NormalTok{artesanal }\OtherTok{\textless{}{-}} \FunctionTok{ifelse}\NormalTok{(beer}\SpecialCharTok{$}\NormalTok{marca }\SpecialCharTok{==} \StringTok{"SAMUEL ADAMS"} \SpecialCharTok{|}\NormalTok{ beer}\SpecialCharTok{$}\NormalTok{marca }\SpecialCharTok{==} \StringTok{"STROH\textquotesingle{}S"} \SpecialCharTok{|}
\NormalTok{                         beer}\SpecialCharTok{$}\NormalTok{marca }\SpecialCharTok{==} \StringTok{"AUGSBURGER"} \SpecialCharTok{|}\NormalTok{ beer}\SpecialCharTok{$}\NormalTok{marca }\SpecialCharTok{==} \StringTok{"PETE\textquotesingle{}S WICKED"} \SpecialCharTok{|}
\NormalTok{                         beer}\SpecialCharTok{$}\NormalTok{marca }\SpecialCharTok{==} \StringTok{"LEINENKUGEL"} \SpecialCharTok{|}\NormalTok{ beer}\SpecialCharTok{$}\NormalTok{marca }\SpecialCharTok{==} \StringTok{"GOOSE ISLAND"} \SpecialCharTok{|}
\NormalTok{                         beer}\SpecialCharTok{$}\NormalTok{marca }\SpecialCharTok{==} \StringTok{"LOWENBRAU"} \SpecialCharTok{|}\NormalTok{ beer}\SpecialCharTok{$}\NormalTok{marca }\SpecialCharTok{==} \StringTok{"OREGON"} \SpecialCharTok{|}
\NormalTok{                         beer}\SpecialCharTok{$}\NormalTok{marca }\SpecialCharTok{==} \StringTok{"BLUE MOON"} \SpecialCharTok{|}\NormalTok{ beer}\SpecialCharTok{$}\NormalTok{marca }\SpecialCharTok{==} \StringTok{"KILLIAN\textquotesingle{}S"} \SpecialCharTok{|}
\NormalTok{                         beer}\SpecialCharTok{$}\NormalTok{marca }\SpecialCharTok{==} \StringTok{"ROGUE"} \SpecialCharTok{|}\NormalTok{ beer}\SpecialCharTok{$}\NormalTok{marca }\SpecialCharTok{==} \StringTok{"BERGHOFF"} \SpecialCharTok{|}
\NormalTok{                         beer}\SpecialCharTok{$}\NormalTok{marca }\SpecialCharTok{==} \StringTok{"GUINNESS STOUT"} \SpecialCharTok{|}\NormalTok{ beer}\SpecialCharTok{$}\NormalTok{marca }\SpecialCharTok{==} \StringTok{"SHIPYARD"} \SpecialCharTok{|}
\NormalTok{                         beer}\SpecialCharTok{$}\NormalTok{marca }\SpecialCharTok{==} \StringTok{"NEW AMSTERDAM"} \SpecialCharTok{|}\NormalTok{ beer}\SpecialCharTok{$}\NormalTok{marca }\SpecialCharTok{==} \StringTok{"CELIS"} \SpecialCharTok{|}
\NormalTok{                         beer}\SpecialCharTok{$}\NormalTok{marca }\SpecialCharTok{==} \StringTok{"PYRAMID"} \SpecialCharTok{|}\NormalTok{ beer}\SpecialCharTok{$}\NormalTok{marca }\SpecialCharTok{==} \StringTok{"ROLLING ROCK"} \SpecialCharTok{|}
\NormalTok{                         beer}\SpecialCharTok{$}\NormalTok{marca }\SpecialCharTok{==} \StringTok{"ANCHOR"} \SpecialCharTok{|}\NormalTok{ beer}\SpecialCharTok{$}\NormalTok{marca }\SpecialCharTok{==} \StringTok{"BADERBRAU"} \SpecialCharTok{|}
\NormalTok{                         beer}\SpecialCharTok{$}\NormalTok{marca }\SpecialCharTok{==} \StringTok{"BLACK DOG"} \SpecialCharTok{|}\NormalTok{ beer}\SpecialCharTok{$}\NormalTok{marca }\SpecialCharTok{==} \StringTok{"BOULDER"} \SpecialCharTok{|}
\NormalTok{                         beer}\SpecialCharTok{$}\NormalTok{marca }\SpecialCharTok{==} \StringTok{"DEVIL"} \SpecialCharTok{|}\NormalTok{ beer}\SpecialCharTok{$}\NormalTok{marca }\SpecialCharTok{==} \StringTok{"DUSSELDORFER"} \SpecialCharTok{|}
\NormalTok{                         beer}\SpecialCharTok{$}\NormalTok{marca }\SpecialCharTok{==} \StringTok{"HACKER PSCHORR"} \SpecialCharTok{|}\NormalTok{ beer}\SpecialCharTok{$}\NormalTok{marca }\SpecialCharTok{==} \StringTok{"NAKED ASPEN"} \SpecialCharTok{|}   
\NormalTok{                         beer}\SpecialCharTok{$}\NormalTok{marca }\SpecialCharTok{==} \StringTok{"POINT"} \SpecialCharTok{|}\NormalTok{ beer}\SpecialCharTok{$}\NormalTok{marca }\SpecialCharTok{==} \StringTok{"SAMUEL SMITH"} \SpecialCharTok{|}  
\NormalTok{                         beer}\SpecialCharTok{$}\NormalTok{marca }\SpecialCharTok{==} \StringTok{"SIERRA NEVADA"} \SpecialCharTok{|}\NormalTok{ beer}\SpecialCharTok{$}\NormalTok{marca }\SpecialCharTok{==} \StringTok{"STATE STREET"} \SpecialCharTok{|}  
\NormalTok{                         beer}\SpecialCharTok{$}\NormalTok{marca }\SpecialCharTok{==} \StringTok{"COLD SPRING"} \SpecialCharTok{|}\NormalTok{ beer}\SpecialCharTok{$}\NormalTok{marca }\SpecialCharTok{==} \StringTok{"ELK MOUNTAIN"} \SpecialCharTok{|}  
\NormalTok{                         beer}\SpecialCharTok{$}\NormalTok{marca }\SpecialCharTok{==} \StringTok{"LEGACY"} \SpecialCharTok{|}\NormalTok{ beer}\SpecialCharTok{$}\NormalTok{marca }\SpecialCharTok{==} \StringTok{"RED HOOK"} \SpecialCharTok{|}  
\NormalTok{                         beer}\SpecialCharTok{$}\NormalTok{marca }\SpecialCharTok{==} \StringTok{"SUMMIT"} \SpecialCharTok{|}\NormalTok{ beer}\SpecialCharTok{$}\NormalTok{marca }\SpecialCharTok{==} \StringTok{"TOSELLI"} \SpecialCharTok{|}  
\NormalTok{                         beer}\SpecialCharTok{$}\NormalTok{marca }\SpecialCharTok{==} \StringTok{"COOPERS"} \SpecialCharTok{|}\NormalTok{ beer}\SpecialCharTok{$}\NormalTok{marca }\SpecialCharTok{==} \StringTok{"DOM"} \SpecialCharTok{|}  
\NormalTok{                         beer}\SpecialCharTok{$}\NormalTok{marca }\SpecialCharTok{==} \StringTok{"JW DUNDEE\textquotesingle{}S"} \SpecialCharTok{|}\NormalTok{ beer}\SpecialCharTok{$}\NormalTok{marca }\SpecialCharTok{==} \StringTok{"RAZOR\textquotesingle{}S"} \SpecialCharTok{|}  
\NormalTok{                         beer}\SpecialCharTok{$}\NormalTok{marca }\SpecialCharTok{==} \StringTok{"RED RIVER"} \SpecialCharTok{|}\NormalTok{ beer}\SpecialCharTok{$}\NormalTok{marca }\SpecialCharTok{==} \StringTok{"WATNEY\textquotesingle{}S"}\NormalTok{ , }\DecValTok{1}\NormalTok{, }\DecValTok{0}\NormalTok{)}

\CommentTok{\# Exportando la base de datos}
\NormalTok{beer2 }\OtherTok{\textless{}{-}} \FunctionTok{subset}\NormalTok{(beer, }\AttributeTok{select =} \FunctionTok{c}\NormalTok{(}\DecValTok{1}\SpecialCharTok{:}\DecValTok{8}\NormalTok{, }\DecValTok{10}\NormalTok{, }\DecValTok{11}\NormalTok{))}
\FunctionTok{write.csv}\NormalTok{(beer2 , }\AttributeTok{file =} \StringTok{"beer2.csv"}\NormalTok{, }\AttributeTok{row.names =} \ConstantTok{FALSE}\NormalTok{)}
\end{Highlighting}
\end{Shaded}

\hypertarget{e.-ejecute-la-misma-regresiuxf3n-del-punto-anterior-utilizando-el-comando-correspondiente-de-r.-reporte-los-estimados-en-una-tabla-en-la-que-incluya-los-resultados-del-literal-anterior-interprete-los-coeficientes-y-compare.-hay-alguna-diferencia-entre-el-procedimiento-manual-y-el-automuxe1tico}{%
\subsubsection{e. Ejecute la misma regresión del punto anterior
utilizando el comando correspondiente de R. Reporte los estimados en una
tabla en la que incluya los resultados del literal anterior, interprete
los coeficientes y compare. ¿Hay alguna diferencia entre el
procedimiento manual y el
automático?}\label{e.-ejecute-la-misma-regresiuxf3n-del-punto-anterior-utilizando-el-comando-correspondiente-de-r.-reporte-los-estimados-en-una-tabla-en-la-que-incluya-los-resultados-del-literal-anterior-interprete-los-coeficientes-y-compare.-hay-alguna-diferencia-entre-el-procedimiento-manual-y-el-automuxe1tico}}

\begin{Shaded}
\begin{Highlighting}[]
\CommentTok{\# Procedimiento automático}
\NormalTok{reg\_auto }\OtherTok{\textless{}{-}} \FunctionTok{lm}\NormalTok{(q }\SpecialCharTok{\textasciitilde{}}\NormalTok{ price }\SpecialCharTok{+}\NormalTok{ importada }\SpecialCharTok{+}\NormalTok{ artesanal , }\AttributeTok{data =}\NormalTok{ beer2)}
\FunctionTok{summary}\NormalTok{(reg\_auto)}
\end{Highlighting}
\end{Shaded}

\begin{table}[!ht] \centering 
  \caption{Resultados regresión MCO} 
  \label{} 
\begin{tabular}{@{\extracolsep{5pt}}lcc} 
\\[-1.8ex]\hline 
\hline \\[-1.8ex] 
 & \multicolumn{2}{c}{\textit{Variable dependiente:}} \\ 
\cline{2-3} 
\\[-1.8ex] & \multicolumn{2}{c}{q} \\ 
\\[-1.8ex] & (1) & (2)\\ 
\hline \\[-1.8ex] 
 price & $-$701,311.40$^{***}$ & $-$701,311.40$^{***}$ \\ 
  & (141,464.00) & (141,464.00) \\ 
  & & \\ 
 importada & 72,514.27 & 72,514.27 \\ 
  & (105,876.90) & (105,876.90) \\ 
  & & \\ 
 artesanal & 86,144.85 & 86,144.85 \\ 
  & (91,731.34) & (91,731.34) \\ 
  & & \\ 
 Constante & 730,440.50$^{***}$ & 730,440.50$^{***}$ \\ 
  & (97,565.98) & (97,565.98) \\ 
  & & \\ 
\hline \\[-1.8ex] 
Observaciones & 753 & 753 \\ 
R$^{2}$ & 0.04 & 0.04 \\ 
R$^{2}$ Ajustado & 0.04 & 0.04 \\ 
Residual Std. Error (df = 749) & 993,428.80 & 993,428.80 \\ 
Estadístico F (df = 3; 749) & 10.77$^{***}$ & 10.77$^{***}$ \\ 
\hline 
\hline \\[-1.8ex] 
\textit{Nota:}  & \multicolumn{2}{r}{$^{*}$p$<$0.1; $^{**}$p$<$0.05; $^{***}$p$<$0.01} \\ 
\end{tabular} 
\end{table}

\hypertarget{h.-con-base-en-la-especificaciuxf3n-del-literal-d-haga-una-particiuxf3n-de-la-matriz-de-variables-regresoras-x-x_1x_2-donde-la-matriz-x_1-estuxe9-compuesta-por-la-variable-price-y-la-matriz-x_2-estuxe9-compuesta-por-las-demuxe1s-variables-regresoras.-muestre-que-en-este-caso-el-efecto-parcial-de-la-variable-price-sobre-eqx_1x_2-se-puede-recuperar-siguiendo-el-teorema-de-frisch-waugh.-describa-esquemuxe1ticamente-los-pasos-que-usted-sigue-al-aplicar-el-teorema-y-explique-la-intuiciuxf3n-asociada-a-cada-uno-de-dichos-pasos.}{%
\subsubsection{\texorpdfstring{h. Con base en la especificación del
literal (d), haga una partición de la matriz de variables regresoras
\(X = [X_{1}X_{2}]\) donde la matriz \(X_{1}\) esté compuesta por la
variable price y la matriz X\_\{2\} esté compuesta por las demás
variables regresoras. Muestre que, en este caso, el efecto parcial de la
variable price sobre \(E[q|X_{1},X_{2}]\) se puede recuperar siguiendo
el Teorema de Frisch-Waugh. Describa esquemáticamente los pasos que
usted sigue al aplicar el Teorema y explique la intuición asociada a
cada uno de dichos
pasos.}{h. Con base en la especificación del literal (d), haga una partición de la matriz de variables regresoras X = {[}X\_\{1\}X\_\{2\}{]} donde la matriz X\_\{1\} esté compuesta por la variable price y la matriz X\_\{2\} esté compuesta por las demás variables regresoras. Muestre que, en este caso, el efecto parcial de la variable price sobre E{[}q\textbar X\_\{1\},X\_\{2\}{]} se puede recuperar siguiendo el Teorema de Frisch-Waugh. Describa esquemáticamente los pasos que usted sigue al aplicar el Teorema y explique la intuición asociada a cada uno de dichos pasos.}}\label{h.-con-base-en-la-especificaciuxf3n-del-literal-d-haga-una-particiuxf3n-de-la-matriz-de-variables-regresoras-x-x_1x_2-donde-la-matriz-x_1-estuxe9-compuesta-por-la-variable-price-y-la-matriz-x_2-estuxe9-compuesta-por-las-demuxe1s-variables-regresoras.-muestre-que-en-este-caso-el-efecto-parcial-de-la-variable-price-sobre-eqx_1x_2-se-puede-recuperar-siguiendo-el-teorema-de-frisch-waugh.-describa-esquemuxe1ticamente-los-pasos-que-usted-sigue-al-aplicar-el-teorema-y-explique-la-intuiciuxf3n-asociada-a-cada-uno-de-dichos-pasos.}}

Teniendo en cuenta la partición del enunciado, nuestro modelo se puede
reescribir como:

\[
{q}= X_{1}\beta_{1}+X_{2}\beta_{2}+\varepsilon 
\] donde \(X_{1}\) es una matriz de dimensión \(n\times1\) que contine
únicamente la variable \emph{price} y \(X_{2}\) es una matriz de
dimensión \(n\times3\) que está compuesta por una columna de unos (de
tal forma que el modelo tenga una constante) y las variables
\emph{importada} y \emph{artesanal}.

En este caso, lo que se busca a través del Teorema de Frisch-Waugh es
recuperar el efecto parcial de la variable \emph{price} a través de 3
pasos que se enlistan a continuación:

\begin{itemize}
\item
  Paso 1: Regresar a \(q\) sobre \(X_{2}\). La regresión sería \[
  q=X_{2}\gamma+\epsilon
  \] y los residuales de esta regresión se guardan como \(e_{2}\). En
  este paso se está calculando qué parte de \(q\) no es explicada por
  \(X_{2}\). En otras palabras, se está limpiando a \(q\) del efecto de
  \(X_{2}\).
\item
  Paso 2: Regresar a \(X_{1}\) sobre \(X_{2}\). La regresión sería \[
  X_{1}=X_{2}\omega+\epsilon
  \] y los residuales de está regresión se guardan como \(e_{21}\).
  Ahora, se está calculando qué parte de \(X_{1}\) no es explicada por
  \(X_{2}\). Es decir, se está limpiando a \(X_{1}\) del efecto de
  \(X_{2}\).
\item
  Paso 3: Regresar los residuales de la primera regresión (\(e_{2}\))
  contra los residuales de la segunda (\(e_{21}\)). La regresión será \[
  e_{2}=e_{21}\beta_{1}+\epsilon
  \] Finalmente, se obtiene que \(\hat{\beta_{1}}\) es el mismo que si
  se hubiera hecho la regresión de \(q\) contra \(X_{1}\) y \(X_{2}\),
  específicamente, \(\hat{\beta_{1}} = -701311.4\).
\end{itemize}

Lo razón de lo anterior es que, al buscar estimar solamente el
coeficiente de la variable \emph{price}, se debe primero remover el
efecto de todas las demás variables regresoras, y de esta forma sí se
podrá obtener el efecto parcial de esta variable sobre \(q\),
Precisamente, los residuales que se incluyen en la regresión del tercer
paso son los que obtuvieron el efecto limpio de \(X_{2}\).

Para terminar de evidenciar esto, escribamos el estimador
\(\hat{\beta_{1}}\) como \[
\hat{\beta_{1}}= [e^{`}_{21} e_{21}]^{-1} e^{`}_{21} e_{2}
\] Adicionalmente sabemos que \(e_{2}=M_{x2} q\) y
\(e_{21}=M_{x2}X_{1}\), donde \(M_{x2}\) es la matriz creadora de
residuales generada por las columnas de \(X_{2}\). Remplazando esto en
el estimador de \(\beta_1\):

\[
\hat{\beta_{1}}=[(M_{x2}X_{1})^{`} M_{x2}X_{1} ]^{-1} (M_{x2}X_{1})^{`} M_{x2} q
\] Distribuyendo la transpuesta: \[
\hat{\beta_{1}}=[X_{1}^{`}M_{x2}^{`} M_{x2}X_{1} ]^{-1} X_{1}^{`}M_{x2}^{`} M_{x2} q
\] Como \(M_{x2}\) es simétrica (\(X_{1}^{`}=X_{1}\)), entonces: \[
\hat{\beta_{1}}=[X_{1}^{`}M_{x2} M_{x2}X_{1} ]^{-1} X_{1}^{`}M_{x2} M_{x2} q
\] Como \(M_{x2}\) es idempotente (\(M_{x2}M_{x2}=M_{x2}\)), entonces:

\[
\hat{\beta_{1}}=[X_{1}^{`} M_{x2} X_{1}]^{-1}[X_{1}^{`} M_{x2} q]
\] donde el primer término está indicando la parte de \(X_{1}\) que no
explica \(X_2\) y el segundo término es la parte de \(q\) que no explica
\(X_2\). Es decir, se está calculando \(\hat{\beta_{1}}\) con unas
matrices a las que se les quitó el efecto de \(X_2\).

\begin{Shaded}
\begin{Highlighting}[]
\NormalTok{beer2 }\OtherTok{\textless{}{-}} \FunctionTok{read\_csv}\NormalTok{(}\StringTok{"beer2.csv"}\NormalTok{)}

\CommentTok{\#{-}{-}{-}{-}{-}{-}{-}{-}{-}{-}{-}{-}{-}{-}{-}{-}{-}{-}{-}{-}{-}{-}{-}{-}{-}{-}{-}{-}{-}{-}{-}{-}{-}{-}{-}{-}{-}{-}\#}
\CommentTok{\#  Aplicación Teorema de Frisch{-}Waugh  \#  }
\CommentTok{\#{-}{-}{-}{-}{-}{-}{-}{-}{-}{-}{-}{-}{-}{-}{-}{-}{-}{-}{-}{-}{-}{-}{-}{-}{-}{-}{-}{-}{-}{-}{-}{-}{-}{-}{-}{-}{-}{-}\#}

\CommentTok{\# Paso 1}
\NormalTok{p1 }\OtherTok{\textless{}{-}} \FunctionTok{lm}\NormalTok{(q }\SpecialCharTok{\textasciitilde{}}\NormalTok{ importada }\SpecialCharTok{+}\NormalTok{ artesanal, }\AttributeTok{data =}\NormalTok{ beer2)}
\NormalTok{beer2}\SpecialCharTok{$}\NormalTok{e2 }\OtherTok{\textless{}{-}} \FunctionTok{residuals}\NormalTok{(p1)}
  \CommentTok{\# Se incluye el intercepto en la regresión porque X2 tiene un vector de unos}

\CommentTok{\# Paso 2}
\NormalTok{p2 }\OtherTok{\textless{}{-}} \FunctionTok{lm}\NormalTok{(price }\SpecialCharTok{\textasciitilde{}}\NormalTok{ importada }\SpecialCharTok{+}\NormalTok{ artesanal, }\AttributeTok{data =}\NormalTok{ beer2)}
\NormalTok{beer2}\SpecialCharTok{$}\NormalTok{e21 }\OtherTok{\textless{}{-}} \FunctionTok{residuals}\NormalTok{(p2) }
  \CommentTok{\#Se incluye el intercepto en la regresión porque X2 tiene un vector de unos }

\CommentTok{\# Paso 3}
\NormalTok{p3 }\OtherTok{\textless{}{-}} \FunctionTok{lm}\NormalTok{(e2 }\SpecialCharTok{\textasciitilde{}}\NormalTok{ e21 }\SpecialCharTok{{-}}\DecValTok{1}\NormalTok{, }\AttributeTok{data =}\NormalTok{ beer2)}
\FunctionTok{summary}\NormalTok{(p3)}
  \CommentTok{\# No se incluye el intercepto porque X1 solo lo constituye la variable *price*}
\end{Highlighting}
\end{Shaded}

\hypertarget{estimaciuxf3n-de-la-elasticidad-precio-de-la-demanda-de-cerveza}{%
\subsection{Estimación de la elasticidad-precio de la demanda de
cerveza}\label{estimaciuxf3n-de-la-elasticidad-precio-de-la-demanda-de-cerveza}}

\hypertarget{a.-haga-un-gruxe1fico-de-dispersiuxf3n-entre-el-logaritmo-de-las-cantidades-vendidas-y-el-logaritmo-del-precio-e-incluya-una-luxednea-de-ajuste-lineal-escriba-el-modelo-de-regresiuxf3n-correspondiente-a-esta-luxednea-de-ajuste-y-comente.-quuxe9-tan-bueno-es-el-ajuste-de-este-modelo-cuxf3mo-se-puede-mejorar-el-ajuste}{%
\subsubsection{a. Haga un gráfico de dispersión entre el logaritmo de
las cantidades vendidas y el logaritmo del precio e incluya una línea de
ajuste lineal, escriba el modelo de regresión correspondiente a esta
línea de ajuste y comente. ¿Qué tan bueno es el ajuste de este modelo?
¿Cómo se puede mejorar el
ajuste?}\label{a.-haga-un-gruxe1fico-de-dispersiuxf3n-entre-el-logaritmo-de-las-cantidades-vendidas-y-el-logaritmo-del-precio-e-incluya-una-luxednea-de-ajuste-lineal-escriba-el-modelo-de-regresiuxf3n-correspondiente-a-esta-luxednea-de-ajuste-y-comente.-quuxe9-tan-bueno-es-el-ajuste-de-este-modelo-cuxf3mo-se-puede-mejorar-el-ajuste}}

\begin{Shaded}
\begin{Highlighting}[]
\NormalTok{beer2}\SpecialCharTok{$}\NormalTok{log\_price }\OtherTok{\textless{}{-}} \FunctionTok{log}\NormalTok{(beer2}\SpecialCharTok{$}\NormalTok{price)}
\NormalTok{beer2}\SpecialCharTok{$}\NormalTok{log\_q }\OtherTok{\textless{}{-}} \FunctionTok{log}\NormalTok{(beer2}\SpecialCharTok{$}\NormalTok{q)}

\NormalTok{reg\_1 }\OtherTok{\textless{}{-}} \FunctionTok{lm}\NormalTok{ (log\_q }\SpecialCharTok{\textasciitilde{}}\NormalTok{ log\_price, }\AttributeTok{data =}\NormalTok{ beer2)}
\FunctionTok{summary}\NormalTok{(reg\_1)}
\end{Highlighting}
\end{Shaded}

\begin{Shaded}
\begin{Highlighting}[]
\FunctionTok{library}\NormalTok{(ggplot2)}
\FunctionTok{ggplot}\NormalTok{(beer2, }\FunctionTok{aes}\NormalTok{(}\AttributeTok{x=}\NormalTok{log\_price, }\AttributeTok{y=}\NormalTok{log\_q)) }\SpecialCharTok{+}
    \FunctionTok{geom\_point}\NormalTok{(}\AttributeTok{shape=}\DecValTok{20}\NormalTok{) }\SpecialCharTok{+}    \CommentTok{\# Use hollow circles}
    \FunctionTok{geom\_smooth}\NormalTok{(}\AttributeTok{method=}\NormalTok{lm, }\AttributeTok{col =} \StringTok{"red"}\NormalTok{, }\AttributeTok{se=}\ConstantTok{FALSE}\NormalTok{) }\SpecialCharTok{+} \FunctionTok{labs}\NormalTok{(}\AttributeTok{y =} \StringTok{"Log(q)"}\NormalTok{, }\AttributeTok{x=} \StringTok{"Log(price)"}\NormalTok{) }\SpecialCharTok{+} \FunctionTok{theme\_bw}\NormalTok{()   }\CommentTok{\# Don\textquotesingle{}t add shaded confidence region}
\end{Highlighting}
\end{Shaded}

\hypertarget{b.-estime-la-elasticidad-precio-de-la-demanda-de-cerveza-mediante-una-regresiuxf3n-lineal-que-incluya-las-siguientes-variables-de-control-una-constante-y-las-caracteruxedsticas-de-la-cerveza-del-literal-c-del-punto-1.1.}{%
\subsubsection{b. Estime la elasticidad precio de la demanda de cerveza
mediante una regresión lineal que incluya las siguientes variables de
control: una constante y las características de la cerveza del literal
(c) del punto
1.1.}\label{b.-estime-la-elasticidad-precio-de-la-demanda-de-cerveza-mediante-una-regresiuxf3n-lineal-que-incluya-las-siguientes-variables-de-control-una-constante-y-las-caracteruxedsticas-de-la-cerveza-del-literal-c-del-punto-1.1.}}

\begin{Shaded}
\begin{Highlighting}[]
\NormalTok{reg\_2 }\OtherTok{\textless{}{-}} \FunctionTok{lm}\NormalTok{(log\_q }\SpecialCharTok{\textasciitilde{}}\NormalTok{ log\_price }\SpecialCharTok{+}\NormalTok{ importada }\SpecialCharTok{+}\NormalTok{ artesanal, }\AttributeTok{data =}\NormalTok{ beer2)}
\FunctionTok{summary}\NormalTok{(reg\_2)}
\end{Highlighting}
\end{Shaded}

\hypertarget{c.-estime-de-nuevo-la-elasticidad-precio-de-la-demanda-de-cerveza-incluyendo-esta-vez-constantes-especuxedficas-por-marca.-reporte-los-resultados-de-esta-regresiuxf3n-junto-con-los-del-literal-anterior-compuxe1relos-y-comente-cuuxe1l-modelo-genera-la-mejor-estimaciuxf3n-de-la-elasticidad-y-por-quuxe9}{%
\subsubsection{c.~Estime de nuevo la elasticidad precio de la demanda de
cerveza incluyendo, esta vez, constantes específicas por marca. Reporte
los resultados de esta regresión junto con los del literal anterior,
compárelos y comente: ¿cuál modelo genera la mejor estimación de la
elasticidad y por
qué?}\label{c.-estime-de-nuevo-la-elasticidad-precio-de-la-demanda-de-cerveza-incluyendo-esta-vez-constantes-especuxedficas-por-marca.-reporte-los-resultados-de-esta-regresiuxf3n-junto-con-los-del-literal-anterior-compuxe1relos-y-comente-cuuxe1l-modelo-genera-la-mejor-estimaciuxf3n-de-la-elasticidad-y-por-quuxe9}}

\begin{table}[!htbp] \centering 
  \caption{} 
  \label{} 
\begin{tabular}{@{\extracolsep{5pt}}lcc} 
\\[-1.8ex]\hline 
\hline \\[-1.8ex] 
 & \multicolumn{2}{c}{\textit{Variable dependiente:}} \\ 
\cline{2-3} 
\\[-1.8ex] & \multicolumn{2}{c}{Log(q)} \\ 
\\[-1.8ex] & (1) & (2)\\ 
\hline \\[-1.8ex] 
 Log(price) & $-$2.506$^{***}$ & $-$1.925$^{***}$ \\ 
  & (0.287) & (0.470) \\ 
  & & \\ 
 Importada & 0.089 & 1.499 \\ 
  & (0.306) & (3.128) \\ 
  & & \\ 
 Artesanal & $-$0.416 & 0.284 \\ 
  & (0.265) & (3.960) \\ 
  & & \\ 
 Constante & 8.222$^{***}$ & 7.677$^{***}$ \\ 
  & (0.222) & (2.798) \\ 
  & & \\ 
\hline \\ [-1.8ex]
Controles de marca &  No &  Sí  \\

\hline \\
[-1.8ex] 
Observaciones & 753 & 753 \\ 
R$^{2}$ & 0.137 & 0.401 \\ 
R$^{2}$ Ajustado & 0.134 & 0.252 \\ 
Residual Std. Error & 3.010 (gl = 749) & 2.797 (gl = 602) \\ 
Estadístico F & 39.677$^{***}$ (gl = 3; 749) & 2.687$^{***}$ (gl = 150; 602) \\ 
\hline 
\hline \\[-1.8ex] 
\textit{Nota:}  & \multicolumn{2}{r}{$^{*}$p$<$0.1; $^{**}$p$<$0.05; $^{***}$p$<$0.01} \\ 
\end{tabular} 
\end{table}

\begin{Shaded}
\begin{Highlighting}[]
\NormalTok{reg\_3 }\OtherTok{\textless{}{-}} \FunctionTok{lm}\NormalTok{(log\_q }\SpecialCharTok{\textasciitilde{}}\NormalTok{ log\_price }\SpecialCharTok{+}\NormalTok{ importada }\SpecialCharTok{+}\NormalTok{ artesanal }\SpecialCharTok{+}\NormalTok{ marca, }\AttributeTok{data =}\NormalTok{ beer2)}
\FunctionTok{summary}\NormalTok{(reg\_3)}
\end{Highlighting}
\end{Shaded}

\hypertarget{simulaciones-de-monte-carlo}{%
\section{2. Simulaciones de Monte
Carlo}\label{simulaciones-de-monte-carlo}}

\hypertarget{a.-simule-150-observaciones-de-una-variable-pseudoaleatoria-x_1-que-se-distribuye-uniforme-en-el-intervalo-0-10-y-150-observaciones-de-una-variable-pseudoaleatoria-v-que-se-distribuye-normal-con-media-2-y-varianza-9.}{%
\subsubsection{\texorpdfstring{a. Simule 150 observaciones de una
variable pseudoaleatoria \(x_{1}\) que se distribuye uniforme en el
intervalo {[}0, 10{]} y 150 observaciones de una variable
pseudoaleatoria \(v\) que se distribuye normal con media 2 y varianza
9.}{a. Simule 150 observaciones de una variable pseudoaleatoria x\_\{1\} que se distribuye uniforme en el intervalo {[}0, 10{]} y 150 observaciones de una variable pseudoaleatoria v que se distribuye normal con media 2 y varianza 9.}}\label{a.-simule-150-observaciones-de-una-variable-pseudoaleatoria-x_1-que-se-distribuye-uniforme-en-el-intervalo-0-10-y-150-observaciones-de-una-variable-pseudoaleatoria-v-que-se-distribuye-normal-con-media-2-y-varianza-9.}}

\begin{Shaded}
\begin{Highlighting}[]
\FunctionTok{remove}\NormalTok{(}\AttributeTok{list =} \FunctionTok{ls}\NormalTok{())}
\FunctionTok{set.seed}\NormalTok{(}\DecValTok{123}\NormalTok{)}

\NormalTok{n }\OtherTok{\textless{}{-}} \DecValTok{150}   \CommentTok{\# n: número de observaciones}

\NormalTok{x1 }\OtherTok{\textless{}{-}} \FunctionTok{runif}\NormalTok{(n, }\DecValTok{0}\NormalTok{, }\DecValTok{10}\NormalTok{)}
\NormalTok{v }\OtherTok{\textless{}{-}} \FunctionTok{rnorm}\NormalTok{(n, }\DecValTok{2}\NormalTok{, }\DecValTok{3}\NormalTok{)}
\end{Highlighting}
\end{Shaded}

\hypertarget{b.-cree-la-variable-x_2-x_1-v.}{%
\subsubsection{\texorpdfstring{b. Cree la variable
\(x_{2} = x_{1} + v\).}{b. Cree la variable x\_\{2\} = x\_\{1\} + v.}}\label{b.-cree-la-variable-x_2-x_1-v.}}

\begin{Shaded}
\begin{Highlighting}[]
\NormalTok{x2 }\OtherTok{\textless{}{-}}\NormalTok{ x1 }\SpecialCharTok{+}\NormalTok{ v}
\end{Highlighting}
\end{Shaded}

\hypertarget{c.-simule-150-observaciones-de-una-variable-aleatoria-varepsilon-que-viene-de-una-distribuciuxf3nn-normal-con-media-0-y-varianza-igual-a-25.-cree-la-variable-y-10-x_1-2x_2-varepsilon.}{%
\subsubsection{\texorpdfstring{c.~Simule 150 observaciones de una
variable aleatoria \(\varepsilon\) que viene de una distribuciónn normal
con media 0 y varianza igual a 25. Cree la variable
\(y = 10 − x_{1} + 2x_{2} + \varepsilon\).}{c.~Simule 150 observaciones de una variable aleatoria \textbackslash varepsilon que viene de una distribuciónn normal con media 0 y varianza igual a 25. Cree la variable y = 10 − x\_\{1\} + 2x\_\{2\} + \textbackslash varepsilon.}}\label{c.-simule-150-observaciones-de-una-variable-aleatoria-varepsilon-que-viene-de-una-distribuciuxf3nn-normal-con-media-0-y-varianza-igual-a-25.-cree-la-variable-y-10-x_1-2x_2-varepsilon.}}

\begin{Shaded}
\begin{Highlighting}[]
\NormalTok{epsilon }\OtherTok{\textless{}{-}} \FunctionTok{rnorm}\NormalTok{(n, }\DecValTok{0}\NormalTok{, }\DecValTok{5}\NormalTok{)}

\NormalTok{y }\OtherTok{=} \DecValTok{10} \SpecialCharTok{{-}}\NormalTok{ x1 }\SpecialCharTok{+}\NormalTok{ (}\DecValTok{2}\SpecialCharTok{*}\NormalTok{x2) }\SpecialCharTok{+}\NormalTok{ epsilon}
\end{Highlighting}
\end{Shaded}

\hypertarget{d.-lleve-a-cabo-una-regresiuxf3n-de-mco-de-y-sobre-x_1-y-x_2-incluyendo-el-intercepto.-quuxe9-tan-bueno-es-el-ajuste}{%
\subsubsection{\texorpdfstring{d.~Lleve a cabo una regresión de MCO de
\(y\) sobre \(x_{1}\) y \(x_{2}\), incluyendo el intercepto. ¿Qué tan
bueno es el
ajuste?}{d.~Lleve a cabo una regresión de MCO de y sobre x\_\{1\} y x\_\{2\}, incluyendo el intercepto. ¿Qué tan bueno es el ajuste?}}\label{d.-lleve-a-cabo-una-regresiuxf3n-de-mco-de-y-sobre-x_1-y-x_2-incluyendo-el-intercepto.-quuxe9-tan-bueno-es-el-ajuste}}

\begin{table}[!htbp] \centering 
  \caption{} 
  \label{} 
\begin{tabular}{@{\extracolsep{5pt}}lc} 
\\[-1.8ex]\hline 
\hline \\[-1.8ex] 
 & \multicolumn{1}{c}{\textit{Dependent variable:}} \\ 
\cline{2-2} 
\\[-1.8ex] & y \\ 
\hline \\[-1.8ex] 
 x1 & $-$0.667$^{***}$ \\ 
  & (0.204) \\ 
  x2 & 1.670$^{***}$ \\ 
  & (0.143) \\ 
  Constant & 11.035$^{***}$ \\ 
  & (0.867) \\ 
 \hline \\[-1.8ex] 
Observations & 150 \\ 
R$^{2}$ & 0.564 \\ 
Adjusted R$^{2}$ & 0.558 \\ 
Residual Std. Error & 4.991 (df = 147) \\ 
F Statistic & 95.029$^{***}$ (df = 2; 147) \\ 
\hline 
\hline \\[-1.8ex] 
\textit{Note:}  & \multicolumn{1}{r}{$^{*}$p$<$0.1; $^{**}$p$<$0.05; $^{***}$p$<$0.01} \\ 
\end{tabular} 
\end{table}

\begin{Shaded}
\begin{Highlighting}[]
\NormalTok{reg }\OtherTok{\textless{}{-}} \FunctionTok{lm}\NormalTok{(y }\SpecialCharTok{\textasciitilde{}}\NormalTok{ x1 }\SpecialCharTok{+}\NormalTok{ x2)}
\FunctionTok{summary}\NormalTok{(reg)}
\end{Highlighting}
\end{Shaded}

\hypertarget{e.-repita-los-pasos-de-los-literales-c-y-d-de-este-experimento-1000-veces-manteniendo-x_1-y-x_2-constantes-y-calcule-las-medias-muestrales-de-los-coeficientes-estimados.-cuxf3mo-es-la-media-de-cada-coeficiente-con-respecto-a-su-respectivo-paruxe1metro-poblacional}{%
\subsubsection{\texorpdfstring{e. Repita los pasos de los literales (c)
y (d) de este experimento 1000 veces manteniendo \(x_{1}\) y \(x_{2}\)
constantes y calcule las medias muestrales de los coeficientes
estimados. ¿Cómo es la media de cada coeficiente con respecto a su
respectivo parámetro
poblacional?}{e. Repita los pasos de los literales (c) y (d) de este experimento 1000 veces manteniendo x\_\{1\} y x\_\{2\} constantes y calcule las medias muestrales de los coeficientes estimados. ¿Cómo es la media de cada coeficiente con respecto a su respectivo parámetro poblacional?}}\label{e.-repita-los-pasos-de-los-literales-c-y-d-de-este-experimento-1000-veces-manteniendo-x_1-y-x_2-constantes-y-calcule-las-medias-muestrales-de-los-coeficientes-estimados.-cuxf3mo-es-la-media-de-cada-coeficiente-con-respecto-a-su-respectivo-paruxe1metro-poblacional}}

\begin{table}[!htbp] \centering 
  \caption{Estadísticas desciptivas experimento 1000 repeticiones} 
  \label{exp1000} 
\begin{tabular}{@{\extracolsep{5pt}}lcccc} 
\\[-1.8ex]\hline 
\hline \\[-1.8ex] 
Statistic & \multicolumn{1}{c}{Media} & \multicolumn{1}{c}{Desv. Est.} & \multicolumn{1}{c}{Mín.} &  \multicolumn{1}{c}{Máx.} \\ 
\hline \\[-1.8ex] 
$\hat{\beta_{0}}$ &  10.021 & 0.887 & 7.140  & 12.863 \\ 
$\hat{\beta_{1}}$ &  $-$1.001 & 0.199 & $-$1.620  & $-$0.324 \\ 
$\hat{\beta_{2}}$ &  1.999 & 0.140 & 1.525  & 2.426 \\ 
\hline \\[-1.8ex] 
\end{tabular} 
\end{table}

\begin{Shaded}
\begin{Highlighting}[]
\NormalTok{r }\OtherTok{\textless{}{-}} \DecValTok{1000}   \CommentTok{\# r: número de repeticiones}

\CommentTok{\# Creamos vectores para guardar las estimaciones en cada repetición}
\NormalTok{betax0 }\OtherTok{\textless{}{-}} \FunctionTok{rep}\NormalTok{(}\DecValTok{0}\NormalTok{, r)}
\NormalTok{betax1 }\OtherTok{\textless{}{-}} \FunctionTok{rep}\NormalTok{(}\DecValTok{0}\NormalTok{, r) }
\NormalTok{betax2 }\OtherTok{\textless{}{-}} \FunctionTok{rep}\NormalTok{(}\DecValTok{0}\NormalTok{, r)}

\ControlFlowTok{for}\NormalTok{ (i }\ControlFlowTok{in} \DecValTok{1}\SpecialCharTok{:}\NormalTok{r)\{}
  
\NormalTok{  epsilon\_i }\OtherTok{\textless{}{-}} \FunctionTok{rnorm}\NormalTok{(n, }\DecValTok{0}\NormalTok{, }\DecValTok{5}\NormalTok{)}
\NormalTok{  y\_i }\OtherTok{=} \DecValTok{10} \SpecialCharTok{{-}}\NormalTok{ x1 }\SpecialCharTok{+}\NormalTok{ (}\DecValTok{2}\SpecialCharTok{*}\NormalTok{x2) }\SpecialCharTok{+}\NormalTok{ epsilon\_i}

\NormalTok{  datos\_i }\OtherTok{\textless{}{-}} \FunctionTok{data.frame}\NormalTok{(}\AttributeTok{Y =}\NormalTok{ y\_i , }\AttributeTok{X1 =}\NormalTok{ x1, }\AttributeTok{X2 =}\NormalTok{ x2, }\AttributeTok{E =}\NormalTok{ epsilon\_i)}
  
\NormalTok{  reg\_i }\OtherTok{\textless{}{-}} \FunctionTok{lm}\NormalTok{( Y }\SpecialCharTok{\textasciitilde{}}\NormalTok{ X1 }\SpecialCharTok{+}\NormalTok{ X2, }\AttributeTok{data =}\NormalTok{ datos\_i)}
  
\NormalTok{  betax0[i] }\OtherTok{\textless{}{-}}\NormalTok{ reg\_i}\SpecialCharTok{$}\NormalTok{coefficients[}\DecValTok{1}\NormalTok{]}
\NormalTok{  betax1[i] }\OtherTok{\textless{}{-}}\NormalTok{ reg\_i}\SpecialCharTok{$}\NormalTok{coefficients[}\DecValTok{2}\NormalTok{]}
\NormalTok{  betax2[i] }\OtherTok{\textless{}{-}}\NormalTok{ reg\_i}\SpecialCharTok{$}\NormalTok{coefficients[}\DecValTok{3}\NormalTok{]}
\NormalTok{\}}

\NormalTok{estimaciones }\OtherTok{\textless{}{-}} \FunctionTok{data.frame}\NormalTok{(}\AttributeTok{BETA0 =}\NormalTok{ betax0, }\AttributeTok{BETA1 =}\NormalTok{ betax1, }\AttributeTok{BETA2 =}\NormalTok{ betax2)}

\FunctionTok{mean}\NormalTok{(estimaciones}\SpecialCharTok{$}\NormalTok{BETA0)}
\FunctionTok{mean}\NormalTok{(estimaciones}\SpecialCharTok{$}\NormalTok{BETA1)}
\FunctionTok{mean}\NormalTok{(estimaciones}\SpecialCharTok{$}\NormalTok{BETA2)}
\end{Highlighting}
\end{Shaded}

\hypertarget{f.-repita-los-pasos-de-los-literales-a-b-y-e-para-n-450-1500-3000.-grafique-la-distribuciuxf3n-de-los-coeficientes-para-cada-tamauxf1o-de-muestra-incluyendo-n-150.-quuxe9-sucede-con-la-distribuciuxf3n-de-los-coeficientes-a-medida-que-el-tamauxf1o-de-muestra-crece-quuxe9-nos-sugiere-este-ejercicio-con-respecto-a-la-normalidad-asintuxf3tica-del-estimador-de-mco}{%
\subsubsection{f.~Repita los pasos de los literales (a), (b) y (e) para
n = 450, 1500, 3000. Grafique la distribución de los coeficientes para
cada tamaño de muestra incluyendo n = 150. ¿Qué sucede con la
distribución de los coeficientes a medida que el tamaño de muestra
crece? ¿Qué nos sugiere este ejercicio con respecto a la normalidad
asintótica del estimador de
MCO?}\label{f.-repita-los-pasos-de-los-literales-a-b-y-e-para-n-450-1500-3000.-grafique-la-distribuciuxf3n-de-los-coeficientes-para-cada-tamauxf1o-de-muestra-incluyendo-n-150.-quuxe9-sucede-con-la-distribuciuxf3n-de-los-coeficientes-a-medida-que-el-tamauxf1o-de-muestra-crece-quuxe9-nos-sugiere-este-ejercicio-con-respecto-a-la-normalidad-asintuxf3tica-del-estimador-de-mco}}

\begin{Shaded}
\begin{Highlighting}[]
\CommentTok{\# Creando lista para guardar los resultados para cada tamaño de muestra}
\NormalTok{result }\OtherTok{\textless{}{-}} \FunctionTok{vector}\NormalTok{(}\StringTok{"list"}\NormalTok{,}\DecValTok{3}\NormalTok{)}
\NormalTok{result}

\ControlFlowTok{for}\NormalTok{ (j }\ControlFlowTok{in} \FunctionTok{c}\NormalTok{(}\DecValTok{450}\NormalTok{, }\DecValTok{1500}\NormalTok{, }\DecValTok{3000}\NormalTok{))\{}
  
\CommentTok{\# Hallando las realizaciones de *x1* y *v* y calculando *x2* para cada tamaño de muestra  }
\NormalTok{  x1\_j }\OtherTok{\textless{}{-}} \FunctionTok{runif}\NormalTok{(j, }\DecValTok{0}\NormalTok{, }\DecValTok{10}\NormalTok{)}
\NormalTok{  v\_j }\OtherTok{\textless{}{-}} \FunctionTok{rnorm}\NormalTok{(j, }\DecValTok{2}\NormalTok{, }\DecValTok{3}\NormalTok{)}
\NormalTok{  x2\_j }\OtherTok{\textless{}{-}}\NormalTok{ x1\_j }\SpecialCharTok{+}\NormalTok{ v\_j}
  
\CommentTok{\# Definiendo el número de repeticiones y creando vectores para guardar la}
\NormalTok{  r }\OtherTok{\textless{}{-}} \DecValTok{1000}
  
\CommentTok{\# Creando vectores para guardar la información de cada repetición  }
\NormalTok{  betax0\_j }\OtherTok{\textless{}{-}} \FunctionTok{rep}\NormalTok{(}\DecValTok{0}\NormalTok{, r)}
\NormalTok{  betax1\_j }\OtherTok{\textless{}{-}} \FunctionTok{rep}\NormalTok{(}\DecValTok{0}\NormalTok{, r) }
\NormalTok{  betax2\_j }\OtherTok{\textless{}{-}} \FunctionTok{rep}\NormalTok{(}\DecValTok{0}\NormalTok{, r)}

\ControlFlowTok{for}\NormalTok{ (i }\ControlFlowTok{in} \DecValTok{1}\SpecialCharTok{:}\NormalTok{r)\{}

\CommentTok{\# En cada repetición se halla una realización de *epsilon* y se calcula *y*    }
\NormalTok{  epsilon\_i }\OtherTok{\textless{}{-}} \FunctionTok{rnorm}\NormalTok{(j, }\DecValTok{0}\NormalTok{, }\DecValTok{5}\NormalTok{)}
\NormalTok{  y\_i }\OtherTok{=} \DecValTok{10} \SpecialCharTok{{-}}\NormalTok{ x1\_j }\SpecialCharTok{+}\NormalTok{ (}\DecValTok{2}\SpecialCharTok{*}\NormalTok{x2\_j) }\SpecialCharTok{+}\NormalTok{ epsilon\_i}

\CommentTok{\# Creando un data frame que contenga las realizaciones de cada repetición}
\NormalTok{  datos\_i }\OtherTok{\textless{}{-}} \FunctionTok{data.frame}\NormalTok{(}\AttributeTok{Y =}\NormalTok{ y\_i , }\AttributeTok{X1 =}\NormalTok{ x1\_j, }\AttributeTok{X2 =}\NormalTok{ x2\_j, }\AttributeTok{E =}\NormalTok{ epsilon\_i)}

\CommentTok{\# Realizando la regresión }
\NormalTok{  reg\_i }\OtherTok{\textless{}{-}} \FunctionTok{lm}\NormalTok{( Y }\SpecialCharTok{\textasciitilde{}}\NormalTok{ X1 }\SpecialCharTok{+}\NormalTok{ X2, }\AttributeTok{data =}\NormalTok{ datos\_i)}
  
\CommentTok{\# Guardando las estimaciones de los coeficientes de la regresión de cada repetición  }
\NormalTok{  betax0\_j[i] }\OtherTok{\textless{}{-}}\NormalTok{ reg\_i}\SpecialCharTok{$}\NormalTok{coefficients[}\DecValTok{1}\NormalTok{]}
\NormalTok{  betax1\_j[i] }\OtherTok{\textless{}{-}}\NormalTok{ reg\_i}\SpecialCharTok{$}\NormalTok{coefficients[}\DecValTok{2}\NormalTok{]}
\NormalTok{  betax2\_j[i] }\OtherTok{\textless{}{-}}\NormalTok{ reg\_i}\SpecialCharTok{$}\NormalTok{coefficients[}\DecValTok{3}\NormalTok{]}
\NormalTok{\}}

\CommentTok{\# Creando data frame que contenga las estimaciones para cada tamaño de muestra  }
\NormalTok{  estimaciones\_j }\OtherTok{\textless{}{-}} \FunctionTok{data.frame}\NormalTok{(}\AttributeTok{BETA0 =}\NormalTok{ betax0\_j, }\AttributeTok{BETA1 =}\NormalTok{ betax1\_j, }\AttributeTok{BETA2 =}\NormalTok{ betax2\_j)}
  
\CommentTok{\# Guardando los resultados para cada tamaño de muestra}
\NormalTok{  result[[j]] }\OtherTok{\textless{}{-}}\NormalTok{ estimaciones\_j}
  
\NormalTok{\}}
\CommentTok{\# Compilando en una sola base las estimaciones de los coeficientes de todas las }
\CommentTok{\# repeticiones y tamaños de muestra}

\NormalTok{  Com }\OtherTok{\textless{}{-}} \FunctionTok{data.frame}\NormalTok{(estimaciones, result[[}\DecValTok{450}\NormalTok{]], result[[}\DecValTok{1500}\NormalTok{]], result[[}\DecValTok{3000}\NormalTok{]])}
\end{Highlighting}
\end{Shaded}

\begin{table}[!htbp] \centering 
  \caption{} 
  \label{} 
\begin{tabular}{@{\extracolsep{5pt}}lccccccc} 
\\[-1.8ex]\hline 
\hline \\[-1.8ex] 
Statistic & \multicolumn{1}{c}{N} & \multicolumn{1}{c}{Mean} & \multicolumn{1}{c}{St. Dev.} & \multicolumn{1}{c}{Min} & \multicolumn{1}{c}{Pctl(25)} & \multicolumn{1}{c}{Pctl(75)} & \multicolumn{1}{c}{Max} \\ 
\hline \\[-1.8ex] 
BETA0 n=150 & 1,000 & 10.021 & 0.887 & 7.140 & 9.396 & 10.630 & 12.863 \\ 
BETA1 n=150 & 1,000 & $-$1.001 & 0.199 & $-$1.620 & $-$1.137 & $-$0.870 & $-$0.324 \\ 
BETA2 n=150& 1,000 & 1.999 & 0.140 & 1.525 & 1.907 & 2.091 & 2.426 \\ 
BETA0.1  n=450& 1,000 & 9.998 & 0.506 & 8.607 & 9.666 & 10.352 & 11.861 \\ 
BETA1.1 n=450& 1,000 & $-$0.992 & 0.110 & $-$1.345 & $-$1.066 & $-$0.917 & $-$0.641 \\ 
BETA2.1 n=450& 1,000 & 1.995 & 0.079 & 1.782 & 1.943 & 2.046 & 2.248 \\ 
BETA0.2 n=1500& 1,000 & 9.994 & 0.271 & 9.160 & 9.812 & 10.186 & 10.756 \\ 
BETA1.2 n=1500& 1,000 & $-$1.000 & 0.059 & $-$1.198 & $-$1.039 & $-$0.959 & $-$0.787 \\ 
BETA2.2 n=1500& 1,000 & 2.000 & 0.041 & 1.877 & 1.972 & 2.030 & 2.118 \\ 
BETA0.3 n=3000& 1,000 & 10.003 & 0.190 & 9.395 & 9.878 & 10.133 & 10.644 \\ 
BETA1.3 n=3000& 1,000 & $-$1.001 & 0.043 & $-$1.146 & $-$1.029 & $-$0.971 & $-$0.852 \\ 
BETA2.3 n=3000& 1,000 & 2.000 & 0.029 & 1.909 & 1.980 & 2.020 & 2.082 \\ 
\hline \\[-1.8ex] 
\end{tabular} 
\end{table}

\includegraphics[width=0.55\textwidth,height=0.55\textheight]{beta0.png}
\includegraphics[width=0.55\textwidth,height=0.55\textheight]{beta1.png}
\includegraphics[width=0.55\textwidth,height=0.55\textheight]{beta2.png}

\hypertarget{referencias}{%
\section{Referencias}\label{referencias}}

Craft Brewer Definition. (s.f.). Brewers Association. Recuperado el día
9 marzo 2021 de
\url{https://www.brewersassociation.org/statistics-and-data/craft-brewer-definition/}

\end{document}
